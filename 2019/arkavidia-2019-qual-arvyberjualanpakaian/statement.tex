\documentclass{article}

\usepackage{geometry}
\usepackage{amsmath}
\usepackage{graphicx}
\usepackage{listings}
\usepackage{hyperref}
\usepackage{multicol}
\usepackage{fancyhdr}
\pagestyle{fancy}
\hypersetup{ colorlinks=true, linkcolor=black, filecolor=magenta, urlcolor=cyan}
\geometry{ a4paper, total={170mm,257mm}, top=20mm, right=20mm, bottom=20mm, left=20mm}
\setlength{\parindent}{0pt}
\setlength{\parskip}{1em}
\renewcommand{\headrulewidth}{0pt}
\lhead{Competitive Programming - Arkavidia V}
\fancyfoot[CE,CO]{\thepage}

\begin{document}

\begin{center}
    \section*{A. Arvy Berjualan Pakaian}

    \begin{tabular}{ | c c | }
        \hline
        Batas Waktu  & 1s \\
        Batas Memori & 64MB \\
        \hline
    \end{tabular}
\end{center}

\subsection*{Deskripsi}

Saat berbelanja, diskon $70\%$ berbeda dengan $50\% + 20\%$.
Misalkan pada sebuah baju yang memiliki harga Rp 100.000, diskon $70\%$ artinya harga dipotong menjadi Rp 30.000, sedangkan diskon $50\% + 20\%$ akan memotong harga sebesar $50\%$ menjadi Rp 50.000, lalu sisanya dipotong $20\%$ menjadi harga akhir Rp 40.000.

Arvy ingin menjual $N$ pakaian bekas.
Untuk menarik pembeli, ia memberikan diskon $X / K$ (diskon tidak dalam bentuk $\%$ namun per-$K$).
Ia menyadari kalau diskon $X / K$ bisa ditulis dengan banyak konfigurasi.
Kini ia bertanya, manakah konfigurasi penulisan diskon yang memiliki jumlah terbesar?
Cara mencari konfigurasi dengan jumlah terbesar adalah sebagai berikut:
\begin{enumerate}
    \item Daftar semua barisan bilangan positif $A_1, A_2, A_3, \dots$ yang menyebabkan diskon $\frac{A_1}{K}, \frac{A_2}{K}, \frac{A_3}{K} , \dots $ sama dengan diskon $X / K$ (dengan cara menghitung yang sudah dijelaskan di atas).
    \item Jika ada lebih dari satu konfigurasi, pilih barisan dengan $A_1 + A_2 + A_3 + \dots $ terbesar.
\end{enumerate}

Perhatikan pula jumlah dari $A_1, A_2, A_3, \dots$ dapat lebih dari 100.

\subsection*{Format Masukan}

Baris pertama terdiri dari satu bilangan bulat positif $T$ ($1 \leq T \leq 10$), menyatakan banyaknya kasus uji.
Untuk tiap kasus uji, baris pertama tediri dari bilangan $N$ ($1 \leq N \leq 100$) dan $K$ ($91 \leq K \leq 100$), menyatakan banyaknya pakaian yang mau dijual dan pembagi diskon seperti deskripsi soal.
$N$ baris berikutnya terdiri dari bilangan $X$ ($1 \leq X < K$), menyatakan diskon yang diinginkan.

\subsection*{Format Keluaran}

Untuk tiap kasus uji, tuliskan bilangan $M$ yang merupakan panjang konfigurasi dengan jumlah terbesar, diikuti dengan $M$ bilangan pembentuk konfigurasi dari kasus uji tersebut.
Jika ada lebih dari satu konfigurasi penulisan diskon dengan jumlah terbesar, keluarkan yang mana saja.
\\

\begin{multicols}{2}
\subsection*{Contoh Masukan}
\begin{lstlisting}
2
2 100
90
9
1 92
70
\end{lstlisting}
\columnbreak
\subsection*{Contoh Keluaran}
\begin{lstlisting}
4 20 50 50 50
1 9
3 46 4 46
\end{lstlisting}
\vfill
\null
\end{multicols}

\pagebreak
\subsection*{Penjelasan}
Pada kasus uji pertama, $90\%$ dapat ditulis ulang menjadi:

\begin{itemize}
    \setlength\itemsep{0pt}
    \item $90\%$
    \item $50\% + 80\%$
    \item $20\% + 50\% + 50\% + 50\%$
    \item dan seterusnya.
\end{itemize}

Dari semua konfigurasi, $20\% + 50\% + 50\% + 50\%$ adalah konfigurasi dengan jumlah terbesar.

\pagebreak

\end{document}