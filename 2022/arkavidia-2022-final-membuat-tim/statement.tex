\documentclass{article}

\usepackage{geometry}
\usepackage{amsmath}
\usepackage{graphicx, eso-pic}
\usepackage{listings}
\usepackage{hyperref}
\usepackage{multicol}
\usepackage{fancyhdr}
\pagestyle{fancy}
\fancyhf{}
\hypersetup{ colorlinks=true, linkcolor=black, filecolor=magenta, urlcolor=cyan}
\geometry{ a4paper, total={170mm,257mm}, top=40mm, right=20mm, bottom=20mm, left=20mm}
\setlength{\parindent}{0pt}
\setlength{\parskip}{0.5em}
\renewcommand{\headrulewidth}{0pt}
\AddToShipoutPictureBG{%
  \AtPageUpperLeft{%
    \raisebox{-\height}{\includegraphics[width=\paperwidth, height=30mm]{../headerarkav.png}}
  }
}
\rfoot{\thepage}
\lfoot{Competitive Programming - Arkavidia 8.0}
\lstset{
    basicstyle=\ttfamily\small,
    columns=fixed,
    extendedchars=true,
    breaklines=true,
    tabsize=2,
    prebreak=\raisebox{0ex}[0ex][0ex]{\ensuremath{\hookleftarrow}},
    frame=none,
    showtabs=false,
    showspaces=false,
    showstringspaces=false,
    prebreak={},
    keywordstyle=\color[rgb]{0.627,0.126,0.941},
    commentstyle=\color[rgb]{0.133,0.545,0.133},
    stringstyle=\color[rgb]{01,0,0},
    captionpos=t,
    escapeinside={(\%}{\%)}
}

\begin{document}

\begin{center}
    \section*{M. Membuat Tim} % ganti judul soal

    \begin{tabular}{ | c c | }
        \hline
        Batas Waktu  & 1s \\    % jangan lupa ganti time limit
        Batas Memori & 512MB \\  % jangan lupa ganti memory limit
        \hline
    \end{tabular}
\end{center}

\subsection*{Deskripsi}
Terdapat sebuah kelas di sekolah Arkavidia yang terdiri dari $N$ murid. Guru dari kelas tersebut ingin membagi murid - murid di kelasnya menjadi 2 tim yang beranggotakan $N - K$ murid dan $K$ murid. Murid ke-$i$ memiliki tingkat kepintaran $P_i$. Guru tersebut ingin membagi murid-murid tersebut menjadi 2 tim dengan nilai \textit{fairness} seminimal mungkin. Nilai \textit{fairness} didefinisikan sebagai selisih dari jumlah kepintaran murid dari kedua tim itu. Tentukan nilai minimal dari \textit{fairness} yang dapat dicapai oleh guru tersebut.

\subsection*{Format Masukan}
Baris pertama terdiri dari dua bilangan positif $N$ ($2 \leq N \leq 30$) yang menyatakan jumlah murid di kelas dan $K$ ($1 \leq K < N$) yang menyatakan jumlah murid di salah satu kelompok.

Baris kedua terdiri dari $N$ buah bilangan bulat $A_i$ ($ 1 \leq i \leq N$ dan $1 \leq A_i \leq 1000$) yang menyatakan kekuatan murid ke-$i$.

\subsection*{Format Keluaran}
Keluarkan satu bilangan bulat yang menyatakan nilai minimal dari \textit{fairness} yang dapat dicapai oleh guru tersebut.

\begin{multicols}{2}
\subsection*{Contoh Masukan}
\begin{lstlisting}
6 3
4 2 9 8 8 6 
\end{lstlisting}
\columnbreak

\subsection*{Contoh Keluaran}
\begin{lstlisting}
1
\end{lstlisting}
\vfill
\null
\end{multicols}

\subsection*{Penjelasan}
Pada contoh masukan, kelompok $N-K$ terdiri atas murid nomor $2, 3$ dan $4$ dengan jumlah kepintarannya adalah 19. Kelompok $N$ terdiri atas murid nomor $1, 5$ dan $6$ dengan jumlah kepintarannya adalah 18. Nilai \textit{fairness} adalah 1. Dapat dibuktikan bahwa nilai \textit{fairness} tidak bisa lebih kecil dari 1.  

\end{document}