\documentclass{article}

\usepackage{geometry}
\usepackage{amsmath}
\usepackage{graphicx, eso-pic}
\usepackage{listings}
\usepackage{hyperref}
\usepackage{multicol}
\usepackage{fancyhdr}
\pagestyle{fancy}
\fancyhf{}
\hypersetup{ colorlinks=true, linkcolor=black, filecolor=magenta, urlcolor=cyan}
\geometry{ a4paper, total={170mm,257mm}, top=40mm, right=20mm, bottom=20mm, left=20mm}
\setlength{\parindent}{0pt}
\setlength{\parskip}{0.5em}
\renewcommand{\headrulewidth}{0pt}
\AddToShipoutPictureBG{%
  \AtPageUpperLeft{%
    \raisebox{-\height}{\includegraphics[width=\paperwidth, height=30mm]{../headerarkav.png}}
  }
}
\rfoot{\thepage}
\lfoot{Competitive Programming - Arkavidia 8.0}
\lstset{
    basicstyle=\ttfamily\small,
    columns=fixed,
    extendedchars=true,
    breaklines=true,
    tabsize=2,
    prebreak=\raisebox{0ex}[0ex][0ex]{\ensuremath{\hookleftarrow}},
    frame=none,
    showtabs=false,
    showspaces=false,
    showstringspaces=false,
    prebreak={},
    keywordstyle=\color[rgb]{0.627,0.126,0.941},
    commentstyle=\color[rgb]{0.133,0.545,0.133},
    stringstyle=\color[rgb]{01,0,0},
    captionpos=t,
    escapeinside={(\%}{\%)}
}

\begin{document}

\begin{center}
    \section*{B. Bangun Segitiga} % ganti judul soal

    \begin{tabular}{ | c c | }
        \hline
        Batas Waktu  & 1s \\    % jangan lupa ganti time limit
        Batas Memori & 256MB \\  % jangan lupa ganti memory limit
        \hline
    \end{tabular}
\end{center}

\subsection*{Deskripsi}
Tuan Arvy merupakan orang yang suka memberikan tantangan. Setelah banyak memberikan tantangan, diapun kehabisan ide untuk tantangan terbarunya. Untuk mendapatkan inspirasi, dia memutuskan untuk pergi ke sebuah pameran seni abstrak di kotanya. Di sana ia melihat sebuah lukisan yang terdiri dari banyak sekali garis. Melihat lukisan ini, iapun mendapatkan ide untuk tantangan terbarunya dan dia meminta Arka untuk menjadi pencoba pertama.

Tuan Arvy akan memberikan Arka $N$ buah garis pada sebuah bidang kartesius dua dimensi dalam bentuk $y = M_ix + C_i$. Tuan Arvy meminta Arka untuk menentukan banyak segitiga maksimum yang dapat dibentuk dengan salah satu sisinya merupakan salah satu garis baru yang dapat dibentuk dimanapun. Agar tidak mengecewakan Tuan Arvy, selesaikanlah tantangan yang ia berikan! Dapat dipastikan bahwa tidak ada garis yang berimpit.

\subsection*{Format Masukan}
Baris pertama terdiri dari satu bilangan bulat positif $N$ ($1 \leq N \leq 10^5$), menyatakan banyaknya garis yang ada pada bidang.

$N$ baris berikutnya terdiri dua bilangan bulat $M_i$ dan $C_i$ ($-10^9 \leq M_i,C_i \leq 10^9$ dan $1 \leq i \leq N$), menyatakan persamaan garis ke-$i$.

\subsection*{Format Keluaran}
Satu buah bilangan bulat yang menyatakan banyak segitiga maksimum yang dapat dibentuk sesuai deskripsi soal.

\begin{multicols}{2}
\subsection*{Contoh Masukan}
\begin{lstlisting}
4
1 3
5 -10
-3 4
3 0
\end{lstlisting}
\columnbreak
\subsection*{Contoh Keluaran}
\begin{lstlisting}
6
\end{lstlisting}
\vfill
\null
\end{multicols}

\subsection*{Penjelasan}
Pada contoh testcase, Arka dapat membentuk 6 segitiga dengan salah satu sisinya adalah garis $y = 2x$. Segitiga tersebut adalah

\begin{enumerate}
\item Segitiga dengan sisi $y = 2x$, $y = x + 3$, dan $y = 5x - 10$.
\item Segitiga dengan sisi $y = 2x$, $y = x + 3$, dan $y = -3x + 4$.
\item Segitiga dengan sisi $y = 2x$, $y = x + 3$, dan $y = 3x$.
\item Segitiga dengan sisi $y = 2x$, $y = 5x - 10$, dan $y = -3x + 4$.
\item Segitiga dengan sisi $y = 2x$, $y = 5x - 10$, dan $y = 3x$.
\item Segitiga dengan sisi $y = 2x$, $y = -3x + 4$, dan $y = 3x$.
\end{enumerate}

\end{document}