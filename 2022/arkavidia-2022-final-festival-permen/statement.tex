\documentclass{article}

\usepackage{geometry}
\usepackage{amsmath}
\usepackage{graphicx, eso-pic}
\usepackage{listings}
\usepackage{hyperref}
\usepackage{multicol}
\usepackage{fancyhdr}
\pagestyle{fancy}
\fancyhf{}
\hypersetup{ colorlinks=true, linkcolor=black, filecolor=magenta, urlcolor=cyan}
\geometry{ a4paper, total={170mm,257mm}, top=40mm, right=20mm, bottom=20mm, left=20mm}
\setlength{\parindent}{0pt}
\setlength{\parskip}{0.5em}
\renewcommand{\headrulewidth}{0pt}
\AddToShipoutPictureBG{%
  \AtPageUpperLeft{%
    \raisebox{-\height}{\includegraphics[width=\paperwidth, height=30mm]{../headerarkav.png}}
  }
}
\rfoot{\thepage}
\lfoot{Competitive Programming - Arkavidia 8.0}
\lstset{
    basicstyle=\ttfamily\small,
    columns=fixed,
    extendedchars=true,
    breaklines=true,
    tabsize=2,
    prebreak=\raisebox{0ex}[0ex][0ex]{\ensuremath{\hookleftarrow}},
    frame=none,
    showtabs=false,
    showspaces=false,
    showstringspaces=false,
    prebreak={},
    keywordstyle=\color[rgb]{0.627,0.126,0.941},
    commentstyle=\color[rgb]{0.133,0.545,0.133},
    stringstyle=\color[rgb]{01,0,0},
    captionpos=t,
    escapeinside={(\%}{\%)}
}

\begin{document}

\begin{center}
    \section*{F. Festival Permen} % ganti judul soal

    \begin{tabular}{ | c c | }
        \hline
        Batas Waktu  & 2s \\    % jangan lupa ganti time limit
        Batas Memori & 512MB \\  % jangan lupa ganti memory limit
        \hline
    \end{tabular}
\end{center}

\subsection*{Deskripsi}
Sebuah pabrik permen akan mengadakan Festival Permen selama $N$ hari ke depan. Pada hari ke-$i$ akan dijual bingkisan permen berisi $P_i$ permen dengan harga $C_i$. Namun karena keterbatasan stok permen, pabrik permen tersebut memberikan aturan sebagai berikut:

\begin{enumerate}
    \setlength\itemsep{0pt}
    \item Setiap orang hanya bisa membeli bingkisan permen \textbf{maksimal sekali dalam satu hari}.
    \item Jika selama $K$ hari ke belakang seseorang sudah membeli $b$ bingkisan permen (dengan $b > 0$), maka harga pembelian permen di hari tersebut dikalikan dengan $b$. Namun jika selama $K$ hari kebelakang ia tidak membeli bingkisan permen, maka harga pembelian permen di hari tersebut tetap.
\end{enumerate}

Pak Arvy sangat menyukai permen.Karena Pak Arvy sangat menyukai permen, Pak Arvy akan mengumpulkan sebanyak-banyaknya permen yang bisa dia dapat. Namun, Pak Arvy hanya memiliki uang sebanyak $U$.
Mari bantu Pak Arvy menentukan jumlah permen terbanyak yang bisa ia dapatkan!

\subsection*{Format Masukan}
Baris pertama terdiri dari tiga bilangan bulat positif $N$ ($1 \leq N \leq 1000$), $U$ ($1 \leq U \leq 1000$), dan $K$ ($1 \leq K \leq 5$), menyatakan jumlah hari, jumlah uang Pak Arvy, dan jumlah hari yang akan dicek untuk aturan kedua.

$N$ baris berikutnya terdiri dari dua bilangan bulat positif, dengan baris ke-$i$ menyatakan bilangan $P_i$ ($1 \leq P_i \leq 10^9$) dan $C_i$ ($1 \leq C_i \leq 100$), menyatakan banyak permen dalam bingkisan dan harga bingkisan pada hari ke-$i$.

\subsection*{Format Keluaran}
Keluarkan satu baris terdiri dari satu bilangan bulat X, menyatakan jumlah permen terbanyak yang bisa didapatkan Pak Arvy.

\begin{multicols}{2}
\subsection*{Contoh Masukan 1}
\begin{lstlisting}
5 50 2
1 100
2 30
1 20
3 30
7 20
\end{lstlisting}
\columnbreak

\subsection*{Contoh Keluaran 1}
\begin{lstlisting}
10
\end{lstlisting}
\vfill
\null
\end{multicols}

\begin{multicols}{2}
\subsection*{Contoh Masukan 2}
\begin{lstlisting}
10 50 5
1 1
2 2
3 3
4 4
5 5
6 6
7 7
8 8
9 9
10 10
\end{lstlisting}
\columnbreak

\subsection*{Contoh Keluaran 2}
\begin{lstlisting}
33
\end{lstlisting}
\vfill
\null
\end{multicols}

\subsection*{Penjelasan}
Pada contoh masukan pertama, Pak Arvy akan membeli bingkisan permen pada hari ke-4 dan ke-5. Pada hari ke-4, Pak Arvy membeli bingkisan permen berisi 3 permen. Pada hari ke-5, Pak Arvy membeli bingkisan permen berisi 7 permen. Jadi, Pak Arvy akan mendapatkan 10 permen. Jumlah uang yang harus dikeluarkan Pak Arvy adalah $30 + 20 \times 1 = 50$.

\pagebreak

\end{document}