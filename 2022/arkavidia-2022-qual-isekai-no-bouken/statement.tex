\documentclass{article}

\usepackage{geometry}
\usepackage{amsmath}
\usepackage{graphicx, eso-pic}
\usepackage{listings}
\usepackage{hyperref}
\usepackage{multicol}
\usepackage{fancyhdr}
\pagestyle{fancy}
\fancyhf{}
\hypersetup{ colorlinks=true, linkcolor=black, filecolor=magenta, urlcolor=cyan}
\geometry{ a4paper, total={170mm,257mm}, top=40mm, right=20mm, bottom=20mm, left=20mm}
\setlength{\parindent}{0pt}
\setlength{\parskip}{0.5em}
\renewcommand{\headrulewidth}{0pt}
\AddToShipoutPictureBG{%
  \AtPageUpperLeft{%
    \raisebox{-\height}{\includegraphics[width=\paperwidth, height=30mm]{../headerarkav.png}}
  }
}
\rfoot{\thepage}
\lfoot{Competitive Programming - Arkavidia 8.0}
\lstset{
    basicstyle=\ttfamily\small,
    columns=fixed,
    extendedchars=true,
    breaklines=true,
    tabsize=2,
    prebreak=\raisebox{0ex}[0ex][0ex]{\ensuremath{\hookleftarrow}},
    frame=none,
    showtabs=false,
    showspaces=false,
    showstringspaces=false,
    prebreak={},
    keywordstyle=\color[rgb]{0.627,0.126,0.941},
    commentstyle=\color[rgb]{0.133,0.545,0.133},
    stringstyle=\color[rgb]{01,0,0},
    captionpos=t,
    escapeinside={(\%}{\%)}
}

\begin{document}

\begin{center}
    \section*{I. Isekai no Bouken} % ganti judul soal

    \begin{tabular}{ | c c | }
        \hline
        Batas Waktu  & 2s \\    % jangan lupa ganti time limit
        Batas Memori & 512MB \\  % jangan lupa ganti memory limit
        \hline
    \end{tabular}
\end{center}

\subsection*{Deskripsi}
Di suatu siang bolong, Arvy sedang asik bermain video game. Tanpa sadar, tiba-tiba Arvy masuk ke dalam isekai dan tidak bisa log out.

Agar dapat log out dari dalam isekai, Arvy perlu menyelesaikan misi berikut: 

\begin{quote}
"Kamu sedang berdiri di suatu garis dengan titik awal 0 dan tujuanmu berada di titik akhir $K$. Di antara kedua titik tersebut, terdapat $N$ portal teleportasi dimana setiap portal teleportasi ke-$i$ terletak padatitik $A_i$. Terdapat 2 cara kamu dapat berpindah:
\begin{enumerate}
    \setlength\itemsep{0pt}
    \item Berjalan, membutuhkan waktu 1 detik dan berpindah dari titik $P$ menuju titik $P+1$.
    \item Teleportasi, membutuhkan waktu $X$ detik dan berpindah dari posisi portal teleportasi ke-$i$ menuju posisi portal teleportasi ke-$(i+L)$ atau posisi portal teleportasi ke-$N$ jika $(i+L)$ $\ge$ $N$.
\end{enumerate}
Waktu yang diberikan untuk menyelesaikan misi adalah $Y$ detik dan Anda dapat menggunakan teleportasi maksimal sebanyak $Z$ kali. 

Selamat berpetualang!!!"
\end{quote}

Tentukan jumlah minimum portal teleportasi yang harus Arvy gunakan agar dapat menyelesaikan misi tanpa melewati batas waktu! Atau keluarkan $-1$ jika tidak mungkin dapat menyelesaikan misi!"

\subsection*{Format Masukan}
Baris pertama terdiri dari 6 bilangan bulat $N$ ($1 \leq N \leq 100.000$), $K$ ($1 \leq K \leq 10^9$), $L$ ($1 \leq L \leq 10^5$), $X$ ($1 \leq X \leq 10^9$), $Y$ ($1 \leq Y \leq 10^9$), dan Z ($1 \leq Z \leq 300$).

Baris kedua terdiri atas $N$ bilangan bulat $A_i$, menyatakan posisi portal teleportasi ke-$i$ terurut membesar ($0$ $<$ $A_1$ $<$ $A_2$ $<$ ... $<$ $A_N$ $\leq$ $K$).

\subsection*{Format Keluaran}
Keluarkan jumlah minimum portal teleportasi yang harus digunakan, atau jika tidak mungkin keluarkan $-1$.

\begin{multicols}{2}
\subsection*{Contoh Masukan}
\begin{lstlisting}
6 100 2 5 50 6
5 14 40 60 61 90
\end{lstlisting}
\columnbreak

\subsection*{Contoh Keluaran}
\begin{lstlisting}
2
\end{lstlisting}
\vfill
\null
\end{multicols}

 \subsection*{Penjelasan}
Pada contoh testcase, dapat digunakan portal 2 dan 4 sehingga total waktu yang dibutuhkan menjadi $14+5+5+10=34$. Jika hanya digunakan satu teleportasi misi tidak mungkin bisa diselesaikan dalam waktu tidak lebih dari 50 detik

\end{document}