\documentclass{article}

\usepackage{geometry}
\usepackage{amsmath}
\usepackage{graphicx, eso-pic}
\usepackage{listings}
\usepackage{hyperref}
\usepackage{multicol}
\usepackage{fancyhdr}
\pagestyle{fancy}
\fancyhf{}
\hypersetup{ colorlinks=true, linkcolor=black, filecolor=magenta, urlcolor=cyan}
\geometry{ a4paper, total={170mm,257mm}, top=40mm, right=20mm, bottom=20mm, left=20mm}
\setlength{\parindent}{0pt}
\setlength{\parskip}{0.5em}
\renewcommand{\headrulewidth}{0pt}
\AddToShipoutPictureBG{%
  \AtPageUpperLeft{%
    \raisebox{-\height}{\includegraphics[width=\paperwidth, height=30mm]{../headerarkav.png}}
  }
}
\rfoot{\thepage}
\lfoot{Competitive Programming - Arkavidia 8.0}
\lstset{
    basicstyle=\ttfamily\small,
    columns=fixed,
    extendedchars=true,
    breaklines=true,
    tabsize=2,
    prebreak=\raisebox{0ex}[0ex][0ex]{\ensuremath{\hookleftarrow}},
    frame=none,
    showtabs=false,
    showspaces=false,
    showstringspaces=false,
    prebreak={},
    keywordstyle=\color[rgb]{0.627,0.126,0.941},
    commentstyle=\color[rgb]{0.133,0.545,0.133},
    stringstyle=\color[rgb]{01,0,0},
    captionpos=t,
    escapeinside={(\%}{\%)}
}

\begin{document}

\begin{center}
    \section*{K. Kepulauan Arkavnesia} % ganti judul soal

    \begin{tabular}{ | c c | }
        \hline
        Batas Waktu  & 1s \\    % jangan lupa ganti time limit
        Batas Memori & 512MB \\  % jangan lupa ganti memory limit
        \hline
    \end{tabular}
\end{center}

\subsection*{Deskripsi}
Vidia adalah seorang menteri pariwisata di sebuah negara kepulauan Arkavnesia. Terdapat $N$ pulau dan $M$ jembatan di Arkavnesia. Tidak ada biaya yang harus dibayarkan jika seseorang ingin menggunakan jembatan tersebut. Akan tetapi, tidak semua pulau terhubungkan oleh jembatan. Oleh karena itu, Vidia memasang mesin teleportasi di setiap pulau.

Mesin teleportasi ini tidak gratis. Pemakaian mesin teleportasi di pulau ke-$i$ harus dibayar sebesar $H_i$. Apabila seseorang teleportasi dari pulau $i$ ke pulau $j$, maka harga yang harus dia bayarkan adalah $H_i + H_j$.

Vidia diberikan Q tugas yang diberikan oleh presiden. Terdapat dua tipe tugas yang diberikan ke Vidia, yaitu:

\begin{enumerate}
    \item Mengubah harga dari suatu pulau
    \item Menghitung ongkos minimum yang dibutuhkan dari pulau $i$ ke pulau $j$ dan memberi tahu hasilnya kepada presiden.
\end{enumerate}
Selesaikan pekerjaaan yang telah diberikan presiden tersebut!

\subsection*{Format Masukan}
Baris pertama terdiri dari tiga bilangan bulat positif $N$ ($1 \leq N \leq 100.000$) yang menyatakan jumlah pulau, $M$ ( $1 \leq M \leq 100.000$) yang menyatakan jumlah jembatan, dan $Q$ ($1 \leq Q \leq 100.000$) yang menyatakan jumlah tugas yang harus diselesaikan.

Baris kedua terdiri dari $N$ buah bilangan positif $H_i$ ($1 \leq i \leq N$ dan $1\leq H_i \leq 10^9$) yang menyatakan harga pemakaian mesin teleportasi pulau ke $i$.

$M$ baris selanjutnya terdiri atas dua bilangan positif yaitu $U$ dan $V$ yang menyatakan bahwa terdapat jembatan di antara pulau $U$ dan pulau $V$ ($1 \leq U,V \leq N$).

$Q$ baris selanjutnya terdiri atas bilangan $T$, $U$, dan $V$ yang menyatakan jenis perintah, dengan perintah sebagai berikut :

\begin{itemize}
    \item Apabila $T$ bernilai 1, maka Vidia ditugaskan untuk mengganti harga di pulau $U$ menjadi $V$.
    \item Apabila $T$ bernilai 2, maka Vidia ditugaskan untuk mencari harga termurah untuk pergi dari pulau $U$ ke pulau $V$.
\end{itemize}

Bisa terdapat jembatan yang dimulai dan berakhir di pulau yang sama. Selain itu, dintara dua pulau bisa terdapat lebih dari 1 jembatan.

\subsection*{Format Keluaran}
Keluarkan harga termurah pada setiap tugas $T$ bernilai 2.

\pagebreak
\begin{multicols}{2}
\subsection*{Contoh Masukan}
\begin{lstlisting}
5 2 4
1 2 3 4 5
3 4
4 5
2 1 3
1 5 2
2 5 1
2 3 5
\end{lstlisting}
\columnbreak
\subsection*{Contoh Keluaran}
\begin{lstlisting}
4
3
0
\end{lstlisting}
\vfill
\null
\end{multicols}

\subsection*{Penjelasan}
Pada contoh testcase, berikut adalah penjelasan untuk setiap tugas yang diberikan presiden:

Pada tugas pertama, Menghitung ongkos minimum yang dibutuhkan dari pulau 1 ke pulau 3. Ongkos minimum tersebut adalah menggunakan mesin teleportasi dari pulau 1 ke pulau 3. Harga yang harus dibayarkan adalah $1 + 3 = 4$.

Pada tugas kedua, Mengubah harga dari pulau 5 menjadi 2. Harga dari pulau 5 sebelumnya adalah 5. Setelah diubah menjadi 2, maka harga dari pulau 5 menjadi 2.

Pada tugas ketiga, Menghitung ongkos minimum yang dibutuhkan dari pulau 5 ke pulau 1. Ongkos minimum tersebut adalah menggunakan mesin teleportasi dari pulau 5 ke pulau 1. Harga yang harus dibayarkan adalah $2 + 1 = 3$.

Pada tugas keempat, Menghitung ongkos minimum yang dibutuhkan dari pulau 3 ke pulau 4. Ongkos minimum tersebut adalah menggunakan jembatan dari pulau 3 ke pulau 4 lalu menggunakan jembatan dari pulau 4 ke pulau 5. Harga yang harus dibayarkan adalah $0$.

\end{document}