\documentclass{article}

\usepackage{geometry}
\usepackage{amsmath}
\usepackage{graphicx, eso-pic}
\usepackage{listings}
\usepackage{hyperref}
\usepackage{multicol}
\usepackage{fancyhdr}
\pagestyle{fancy}
\fancyhf{}
\hypersetup{ colorlinks=true, linkcolor=black, filecolor=magenta, urlcolor=cyan}
\geometry{ a4paper, total={170mm,257mm}, top=40mm, right=20mm, bottom=20mm, left=20mm}
\setlength{\parindent}{0pt}
\setlength{\parskip}{0.3em}
\renewcommand{\headrulewidth}{0pt}
\AddToShipoutPictureBG{%
  \AtPageUpperLeft{%
    \raisebox{-\height}{\includegraphics[width=\paperwidth, height=30mm]{../headerarkav.png}}
  }
}
\rfoot{\thepage}
\lfoot{Penyisihan Competitive Programming - Arkavidia 7.0}
\lstset{
    basicstyle=\ttfamily\small,
    columns=fixed,
    extendedchars=true,
    breaklines=true,
    tabsize=2,
    prebreak=\raisebox{0ex}[0ex][0ex]{\ensuremath{\hookleftarrow}},
    frame=none,
    showtabs=false,
    showspaces=false,
    showstringspaces=false,
    prebreak={},
    keywordstyle=\color[rgb]{0.627,0.126,0.941},
    commentstyle=\color[rgb]{0.133,0.545,0.133},
    stringstyle=\color[rgb]{01,0,0},
    captionpos=t,
    escapeinside={(\%}{\%)}
}

\begin{document}

\begin{center}
    \section*{H - Hadiah untuk Kembaran} % ganti judul soal

    \begin{tabular}{ | c c | }
        \hline
        Batas Waktu  & 1s \\    % jangan lupa ganti time limit
        Batas Memori & 256MB \\  % jangan lupa ganti memory limit
        \hline
    \end{tabular}
\end{center}

\subsection*{Deskripsi}

Terdapat $N$ hadiah yang dinomori $1$, $2$, $3$, ..., $N$. Angka favorit si kembar Budi dan Dono berturut-turut adalah $X$ dan $Y$. Setiap orang akan mengambil hadiah yang habis dibagi oleh angka favoritnya. Jika ada hadiah yang habis dibagi oleh kedua angka favoritnya, maka hadiah tersebut tidak akan diambil. Berapakah banyak hadiah yang diambil oleh Budi dan Dono?

\subsection*{Format Masukan}

Baris pertama berisi sebuah bilangan bulat $Q$ ($1 \leq Q \leq 100.000$), menyatakan banyaknya query.
\\
$Q$ baris selanjutnya berisi $3$ bilangan bulat $N$, $X$, $Y$ ($1 \leq N, X, Y \leq 10^9$) berturut-turut menyatakan banyak hadiah yang ada, angka favorit Budi, dan angka favorit Dono.

\subsection*{Format Keluaran}

$Q$ baris, dengan setiap baris berisi dua bilangan bulat, yakni banyak hadiah yang diambil Budi, dan banyak hadiah yang diambil Dono.


\begin{multicols}{2}
\subsection*{Contoh Masukan}
\begin{lstlisting}
3
10 2 3
100 2 4
100 3 3
\end{lstlisting}
\columnbreak
\subsection*{Contoh Keluaran}
\begin{lstlisting}
4 2
25 0
0 0
\end{lstlisting}
\vfill
\null
\end{multicols}

\subsection*{Penjelasan}
Pada query pertama, hadiah yang diambil budi adalah $4$ hadiah yang masing-masing bernomor $2$, $4$, $8$, dan $10$. Kemudian hadiah yang diambil Dono adalah $2$ hadiah yang masing-masing bernomor $3$ dan $9$

\pagebreak

\end{document}