\section*{Problem H - Hadiah untuk Kembaran}
\addcontentsline{toc}{section}{Problem H - Hadiah untuk Kembaran}
\textit{Author: Muhammad Hasan}
\\
\textit{Expected Difficulty: Easy}

Soal ini cukup mudah, kita hanya perlu mengetahui KPK dari nomor favorit budi dan dono untuk mengetahui nomor mana saja yang tidak akan diambil. Menghitung KPK dari dua angka $X$ dan $Y$, dapat dihitung dengan GCD/FPB sebagai berikut:

$$ KPK(X, Y) = \frac{X \times Y}{FPB(X, Y)}$$

Kita akan dapat bahwa jawaban dari soal ini adalah:

Banyaknya hadiah yang diambil Budi $(H_{Budi})$:
\begin{align*}  
    H_{Budi} &= \text{Banyaknya hadiah yang habis dibagi X} - \text{Banyaknya hadiah yang habis dibagi X dan Y} \\
    &= \left \lfloor \frac{N}{X} \right \rfloor - \left \lfloor \frac{N}{KPK(X, Y)} \right \rfloor
\end{align*}

Banyaknya hadiah yang diambil Dono $(H_{Dono})$:
\begin{align*}  
    H_{Dono} &= \text{Banyaknya hadiah yang habis dibagi Y} - \text{Banyaknya hadiah yang habis dibagi X dan Y} \\
    &= \left \lfloor \frac{N}{Y} \right \rfloor - \left \lfloor \frac{N}{KPK(X, Y)} \right \rfloor
\end{align*}
Untuk implementasinya, kita perlu menggunakan
\lstinline|integer 64-bit|, dalam menghitung KPK.

Kode solusi: \url{https://ideone.com/mbQNab}

Kompleksitas Waktu: $O(\log N)$ untuk setiap testcase.

Catatan Panitia: Pembuat soal juga kembar.