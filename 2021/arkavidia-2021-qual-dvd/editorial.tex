\section*{Problem D - DVD}
\addcontentsline{toc}{section}{Problem D - DVD}
\textit{Author: Kinantan Arya Bagaspati}
\\
\textit{Expected Difficulty: Medium}

Alih alih melihat sinar pada soal memantul-mantul pada dinding, cobalah memandang soal sebagai sinar yang lurus,
serta cerminkanlah persegi panjang pada soal (sebutlah sebagai "box") di setiap sisinya infinitely-many 
(sehingga dapat divisualkan sebagai grid of boxes).
Dari hasil cerminan terbentuk 4 jenis box, yakni box awal, horizontal-mirrored-box, vertical-mirrored-box, dan both-mirrored-box.
Perhatikan bahwa dalam grid of boxes yang kita ciptakan, pola jenis box akan berulang tiap $2 \times 2$ box ($2N \times 2M$).
Jika kemudian kita bicara dalam bahasa koordinat dimana entry matrix pada pojok kiri atas box awal ialah $(1,1)$
dan pojok kanan bawah box awal ialah $(N,M)$, maka entry pada koordinat $(x,y)$ akan dicerminkan ke:
horizontal-mirrored-box: $(x, 2M+1-y)$
vertical-mirrored-box: $(2N+1-x, y)$
both-mirrored-box: $(2N+1-x, 2M+1-y)$
Mengingat pola grid of boxes yang sudah ada, entry ini akan dicerminkan di semua koordinat yang berbentuk:
$$(i(2N) + (x \vee (2N+1-x)), j(2M) + (y \vee (2M+1-y)))$$ untuk setiap bilangan bulat $i$ dan $j$

Jelas bahwa sinar akan melewati semua koordinat dengan bentuk $(a,a)$, dan akan berhenti (mengenai sudut) bila $a$ habis dibagi $N$ dan $M$,
yakni saat $a = lcm(N,M)$.
Mudah dibuktikan pula bahwa setiap entry pada matrix dilewati sinar hanya maksimal sekali saja.

\textit{Bukti.}
Kita kembali ke sudut pandang sinar memantul-mantul. Sinar dengan arah kiri-bawah dan kanan-atas hanya menyentuh koordinat $(x,y)$
dengan $x+y$ ganjil, sementara sinar dengan arah kiri-atas dan kanan-bawah hanya menyentuh koordinat $(x,y)$ dengan $x+y$ genap.
Setelah itu cukup jelas bahwa apabila terdapat entry yang terkena sinar 2 kali, jelas bahwa sinar tersebut telah menyentuh sudut.

Jelas pula bahwa bilangan yang tertera pada koordinat $(a,a)$ ialah:

$$\left(\left \lfloor\frac{a-1}{N}\right\rfloor + \left\lfloor\frac{a-1}{M}\right \rfloor\right) \% 9 + 1$$

Maka soal telah di reduce menjadi mencari bilangan bulat i dan j sehingga fakta berikut berlaku:
$$i(2N) + (x \vee (2N+1-x)) = j(2M) + (y \vee (2M+1-y)) \leq lcm(n,m)$$
Untuk itu diperlukan fungsi bernama \textbf{bezout} (sebuah teorema mengenai metode Euclid pada 2 bilangan) yang apabila diinputkan bilangan $2N$ dan $2M$
akan mengembalikan $p$ dan $q$ sehingga $p(2N) + q(2M) = gcd(2N,2M)$
Setelah itu akan dicoba 4 kasus, dimana masing-masing kasusnya mengganti value $U$ menjadi $x$ atau $2N+1-x$ dan $V$ menjadi $y$ atau $2M+1-y$. 
Pada tiap kasus akan dicari nilai $i$ dan $j$ sehingga $i(2N) + (-j)(2M) = V - U$ dengan bantuan bezout.

Akhir kata, semua informasi di atas telah cukup untuk menyelesaikan soal.

Kode Solusi: \url{https://ideone.com/QAERrw}

Kompleksitas Waktu: $O(\log(\max(N, M)))$

Catatan Panitia: Buat yang belum sadar, pola yang dibentuk sinar merupakan pola DVD player screensaver.