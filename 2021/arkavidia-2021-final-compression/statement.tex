\documentclass{article}

\usepackage{geometry}
\usepackage{amsmath}
\usepackage{graphicx, eso-pic}
\usepackage{listings}
\usepackage{hyperref}
\usepackage{multicol}
\usepackage{fancyhdr}
\pagestyle{fancy}
\fancyhf{}
\hypersetup{ colorlinks=true, linkcolor=black, filecolor=magenta, urlcolor=cyan}
\geometry{ a4paper, total={170mm,257mm}, top=40mm, right=20mm, bottom=20mm, left=20mm}
\setlength{\parindent}{0pt}
\setlength{\parskip}{0.3em}
\renewcommand{\headrulewidth}{0pt}
\AddToShipoutPictureBG{%
  \AtPageUpperLeft{%
    \raisebox{-\height}{\includegraphics[width=\paperwidth, height=30mm]{../headerarkav.png}}
  }
}
\rfoot{\thepage}
\lfoot{Final Competitive Programming - Arkavidia 7.0}
\lstset{
    basicstyle=\ttfamily\small,
    columns=fixed,
    extendedchars=true,
    breaklines=true,
    tabsize=2,
    prebreak=\raisebox{0ex}[0ex][0ex]{\ensuremath{\hookleftarrow}},
    frame=none,
    showtabs=false,
    showspaces=false,
    showstringspaces=false,
    prebreak={},
    keywordstyle=\color[rgb]{0.627,0.126,0.941},
    commentstyle=\color[rgb]{0.133,0.545,0.133},
    stringstyle=\color[rgb]{01,0,0},
    captionpos=t,
    escapeinside={(\%}{\%)}
}

\begin{document}

\begin{center}
    \section*{C - Compression} % ganti judul soal

    \begin{tabular}{ | c c | }
        \hline
        Batas Waktu  & 1s \\    % jangan lupa ganti time limit
        Batas Memori & 128MB \\  % jangan lupa ganti memory limit
        \hline
    \end{tabular}
\end{center}

\subsection*{Deskripsi}
Diberikan sebuah string $S$ yang terdiri dari huruf kecil \lstinline|a| sampai \lstinline|z|.

Terdapat sebuah cara untuk melakukan kompresi pada string $S$. Untuk melakukan kompresi pada string $S$ dapat dilakukan dengan menulis huruf kecil \lstinline|a| sampai \lstinline|z|  atau dengan notasi kelipatan substring. Anda diberikan harga untuk menuliskan notasi kelipatan substring dan harga untuk menuliskan masing-masing huruf. Tentukan harga minimum untuk menuliskan string $S$.

Contoh penggunaan notasi kelipatan substring:
\begin{itemize}
\item string \lstinline|b3(ab)b| sama dengan string \lstinline|babababb|
\item string \lstinline|3(c)| sama dengan string \lstinline|ccc|
\end{itemize}

\subsection*{Format Masukan}
Baris pertama berisi bilangan bulat $N$ $(1 \leq N \leq 700)$, menyatakan panjang string $S$

Baris kedua berisi string $S$ yang terdiri dari huruf kecil alfabet.

Baris ketiga berisi bilangan bulat $K$ $(1 \leq K \leq 10^6)$, menyatakan harga untuk menuliskan notasi kelipatan substring.

Baris keempat berisi 26 bilangan bulat $C_1, C_2, \dots, C_{26}$ $(1 \leq C_i \leq 10^6)$, menyatakan harga masing-masing huruf berurutan dari \lstinline|a| sampai \lstinline|z|.

\subsection*{Format Keluaran}
Keluarkan satu baris berisi sebuah bilangan bulat menyatakan harga minimum untuk menuliskan string $S$

\subsection*{Contoh Masukan}
\begin{lstlisting}
13
aaabaaabaaabc
2
1 1 1 1 1 1 1 1 1 1 1 1 1 1 1 1 1 1 1 1 1 1 1 1 1 1
\end{lstlisting}

\subsection*{Contoh Keluaran}
\begin{lstlisting}
7
\end{lstlisting}

\subsection*{Penjelasan}
String $S$ dapat ditulis dengan notasi \lstinline|3(3(a)b)c|. sehingga dibutuhkan harga 1 
untuk menuliskan masing2 huruf \lstinline|a|, \lstinline|b|, dan \lstinline|c|. dan 2 kali penulisan notasi kelipatan.

Sehingga total harganya adalah $1 + 1 + 1 + 2 + 2 = 7$

\end{document}