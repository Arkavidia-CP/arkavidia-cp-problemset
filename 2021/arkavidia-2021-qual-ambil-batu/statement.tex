\documentclass{article}

\usepackage{geometry}
\usepackage{amsmath}
\usepackage{graphicx, eso-pic}
\usepackage{listings}
\usepackage{hyperref}
\usepackage{multicol}
\usepackage{fancyhdr}
\pagestyle{fancy}
\fancyhf{}
\hypersetup{colorlinks=true, linkcolor=black, filecolor=magenta, urlcolor=cyan}
\geometry{ a4paper, total={170mm,257mm}, top=40mm, right=20mm, bottom=20mm, left=20mm}
\setlength{\parindent}{0pt}
\setlength{\parskip}{0.3em}
\renewcommand{\headrulewidth}{0pt}
\AddToShipoutPictureBG{%
  \AtPageUpperLeft{%
    \raisebox{-\height}{\includegraphics[width=\paperwidth, height=30mm]{../headerarkav.png}}
  }
}
\rfoot{\thepage}
\lfoot{Penyisihan Competitive Programming - Arkavidia 7.0}
\lstset{
    basicstyle=\ttfamily\small,
    columns=fixed,
    extendedchars=true,
    breaklines=true,
    tabsize=2,
    prebreak=\raisebox{0ex}[0ex][0ex]{\ensuremath{\hookleftarrow}},
    frame=none,
    showtabs=false,
    showspaces=false,
    showstringspaces=false,
    prebreak={},
    keywordstyle=\color[rgb]{0.627,0.126,0.941},
    commentstyle=\color[rgb]{0.133,0.545,0.133},
    stringstyle=\color[rgb]{01,0,0},
    captionpos=t,
    escapeinside={(\%}{\%)}
}

\begin{document}

\begin{center}
    \section*{A - Ambil Batu} % ganti judul soal

    \begin{tabular}{ | c c | }
        \hline
        Batas Waktu  & 1s \\    % jangan lupa ganti time limit
        Batas Memori & 256MB \\  % jangan lupa ganti memory limit
        \hline
    \end{tabular}
\end{center}

\subsection*{Deskripsi}

Elon dan Melvin sedang bermain sebuah game. Terdapat sebuah tumpukan batu yang berisi $M$ buah batu. Lalu, mereka boleh mengambil sejumlah batu dari tumpukan tersebut. Jumlah yang boleh diambil harus merupakan suatu bilangan dari himpunan $\{A_1, A_2, ..., A_N\}$. Pemain akan dinyatakan kalah bila ia tidak dapat mengambil batu.

Elon akan bergerak pertama kali. Mereka akan memainkan $Q$ buah permainan. Dengan mengasumsikan bahwa setiap pemain akan selalu melakukan gerakan terbaik, tentukan siapa yang akan memenangkan setiap permainan!

\subsection*{Format Masukan}
Baris pertama berisi dua buah bilangan $N$ $(1 \leq N \leq 20)$, menyatakan jumlah bilangan pada himpunan dan $Q$ $(1 \leq Q \leq 10^5)$, menyatakan banyaknya permainan.

Baris kedua berisi $N$ bilangan asli $A_1, A_2, ..., A_N$ $(1 \leq A_i \leq 20)$, berupa himpunan banyaknya batu yang dapat diambil. Dijamin bahwa setiap elemen pada A unik.

$Q$ baris berikutnya berisi satu bilangan bulat $M$ $(1 \leq M \leq 10^{18})$, menyatakan jumlah batu pada tumpukan.\\

\subsection*{Format Keluaran}
$Q$ buah baris yang berisi nama pemenang permainan ("Elon" atau "Melvin" tanpa tanda kutip).

\begin{multicols}{2}
\subsection*{Contoh Masukan}
\begin{lstlisting}
4 5
1 2 3 4
13
26
39
52
65

\end{lstlisting}
\columnbreak
\subsection*{Contoh Keluaran}
\begin{lstlisting}
Elon
Elon
Elon
Elon
Melvin
\end{lstlisting}
\vfill
\null
\end{multicols}

\subsection*{Penjelasan}
Pada permainan pertama, terdapat 13 buah batu. Salah satu skenario yang mungkin terjadi adalah:
\begin{itemize}
\item{Elon mengambil 3 buah batu, sehingga tersisa 10 buah batu.}
\item{Melvin mengambil 2 buah batu, sehingga tersisa 8 buah batu.}
\item{Elon mengambil 3 buah batu, sehingga tersisa 5 buah batu.}
\item{Melvin mengambil 4 buah batu, sehingga tersisa 1 buah batu.}
\item{Elon mengambil 1 buah batu, sehingga tersisa 0 buah batu.}
\item{Melvin tidak dapat mengambil batu.}
\end{itemize}
Oleh karena itu, Elon memenangkan permainan pertama.

\end{document}