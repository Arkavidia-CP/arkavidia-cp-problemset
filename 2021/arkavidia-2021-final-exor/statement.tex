\documentclass{article}

\usepackage{geometry}
\usepackage{amsmath}
\usepackage{graphicx, eso-pic}
\usepackage{listings}
\usepackage{hyperref}
\usepackage{multicol}
\usepackage{fancyhdr}
\pagestyle{fancy}
\fancyhf{}
\hypersetup{ colorlinks=true, linkcolor=black, filecolor=magenta, urlcolor=cyan}
\geometry{ a4paper, total={170mm,257mm}, top=40mm, right=20mm, bottom=20mm, left=20mm}
\setlength{\parindent}{0pt}
\setlength{\parskip}{0.3em}
\renewcommand{\headrulewidth}{0pt}
\AddToShipoutPictureBG{%
  \AtPageUpperLeft{%
    \raisebox{-\height}{\includegraphics[width=\paperwidth, height=30mm]{../headerarkav.png}}
  }
}
\rfoot{\thepage}
\lfoot{Final Competitive Programming - Arkavidia 7.0}
\lstset{
    basicstyle=\ttfamily\small,
    columns=fixed,
    extendedchars=true,
    breaklines=true,
    tabsize=2,
    prebreak=\raisebox{0ex}[0ex][0ex]{\ensuremath{\hookleftarrow}},
    frame=none,
    showtabs=false,
    showspaces=false,
    showstringspaces=false,
    prebreak={},
    keywordstyle=\color[rgb]{0.627,0.126,0.941},
    commentstyle=\color[rgb]{0.133,0.545,0.133},
    stringstyle=\color[rgb]{01,0,0},
    captionpos=t,
    escapeinside={(\%}{\%)}
}

\begin{document}

\begin{center}
    \section*{E - ExOR} % ganti judul soal

    \begin{tabular}{ | c c | }
        \hline
        Batas Waktu  & 1s \\    % jangan lupa ganti time limit
        Batas Memori & 512MB \\  % jangan lupa ganti memory limit
        \hline
    \end{tabular}
\end{center}

\subsection*{Deskripsi}
Arvy memiliki sebuah array sepanjang $N$ yang terdiri dari bilangan bulat positif. Lalu, dia menuliskan \textit{pairwise} XOR dari masing-masing elemen array tersebut di dalam sebuah tabel A berukuran $N \times N$.

Vidia, temannya yang usil lalu menghilangkan beberapa elemen dari tabel tersebut. Untungnya, Arvy masih ingat bahwa array yang ia buat adalah array paling kecil yang mungkin secara leksikografis.

Bantulah Arvy untuk mengembalikan array tersebut, atau beritahu Arvy bahwa array tersebut tidak mungkin ada.

\subsection*{Format Masukan}
Baris pertama berisi sebuah bilangan bulat $N$ $(1 \leq N \leq 500)$, yang menyatakan panjang array.

Baris selanjutnya berisi sebuah matriks berukuran $N \times N$, berisi bilangan bulat $A_{ij}$ $(1 \leq A_{ij} \leq 10^{18})$ untuk setiap $i$ dan $j$ $(1 \leq i, j \leq N)$, dengan elemen yang hilang ditandai dengan bilangan $-1$. 

\subsection*{Format Keluaran}
Jika array tersebut ada, pada baris pertama keluaran berupa string \lstinline|"YES"| (tanpa tanda petik) dilanjutkan dengan array tersebut pada baris berikutnya.

Jika array tersebut tidak ada, keluaran berupa string \lstinline|"NO"| (tanpa tanda petik). 

\begin{multicols}{2}
\subsection*{Contoh Masukan 1}
\begin{lstlisting}
4
0 6 -1 -1
-1 0 -1 5  
11 -1 0 -1 
-1 -1 8 0
\end{lstlisting}
\columnbreak

\subsection*{Contoh Keluaran 1}
\begin{lstlisting}
YES
0 6 11 3
\end{lstlisting}
\vfill
\null
\end{multicols}

\begin{multicols}{2}
\subsection*{Contoh Masukan 2}
\begin{lstlisting}
3
0 2 0
2 0 -1
0 1 0
\end{lstlisting}
\columnbreak

\subsection*{Contoh Keluaran 2}
\begin{lstlisting}
NO
\end{lstlisting}
\vfill
\null
\end{multicols}

\pagebreak

\end{document}