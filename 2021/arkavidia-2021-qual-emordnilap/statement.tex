\documentclass{article}

\usepackage{geometry}
\usepackage{amsmath}
\usepackage{graphicx, eso-pic}
\usepackage{listings}
\usepackage{hyperref}
\usepackage{multicol}
\usepackage{fancyhdr}
\pagestyle{fancy}
\fancyhf{}
\hypersetup{ colorlinks=true, linkcolor=black, filecolor=magenta, urlcolor=cyan}
\geometry{ a4paper, total={170mm,257mm}, top=40mm, right=20mm, bottom=20mm, left=20mm}
\setlength{\parindent}{0pt}
\setlength{\parskip}{0.3em}
\renewcommand{\headrulewidth}{0pt}
\AddToShipoutPictureBG{%
  \AtPageUpperLeft{%
    \raisebox{-\height}{\includegraphics[width=\paperwidth, height=30mm]{../headerarkav.png}}
  }
}
\rfoot{\thepage}
\lfoot{Penyisihan Competitive Programming - Arkavidia 7.0}
\fancyfoot[CE,CO]{\thepage}
\lstset{
    basicstyle=\ttfamily\small,
    columns=fixed,
    extendedchars=true,
    breaklines=true,
    tabsize=2,
    prebreak=\raisebox{0ex}[0ex][0ex]{\ensuremath{\hookleftarrow}},
    frame=none,
    showtabs=false,
    showspaces=false,
    showstringspaces=false,
    prebreak={},
    keywordstyle=\color[rgb]{0.627,0.126,0.941},
    commentstyle=\color[rgb]{0.133,0.545,0.133},
    stringstyle=\color[rgb]{01,0,0},
    captionpos=t,
    escapeinside={(\%}{\%)}
}

\begin{document}

\begin{center}
    \section*{E - Emordnilap} % ganti judul soal

    \begin{tabular}{ | c c | }
        \hline
        Batas Waktu  & 1s \\    % jangan lupa ganti time limit
        Batas Memori & 256MB \\  % jangan lupa ganti memory limit
        \hline
    \end{tabular}
\end{center}

\subsection*{Deskripsi}
Emor adalah orang yang percaya kepada hal yang aneh. Suatu hari, ia percaya bahwa dengan menulis sebuah kata $S$ dengan terbalik, maka ia akan beruntung. Namun, Emor hanya bisa menulis karakter berupa huruf alfabet non-kapital dan angka 0 sampai 9 saja, dan jika ada karakter alfabet kapital maka akan diubah menjadi non-kapital. Jika ada karakter yang tidak bisa ditulis Emor, maka Emor akan menyerah saja. Anda diberikan juga kata $S$ tersebut, tentukan hasil kata yang akan ditulis Emor dan apakah Emor beruntung atau tidak!

\subsection*{Format Masukan}
Satu baris berisi string $S$ $(1 \leq |S| \leq 100)$, menyatakan kata yang diberikan dan terdiri karakter alfabet, karakter angka, atau karakter lain.

\subsection*{Format Keluaran}
Satu baris yang berisi kata yang telah ditulis Emor. Jika Emor tidak bisa menulis dan menyerah, maka keluarkan "Emor tidak beruntung :(" (tanpa tanda petik).

\begin{multicols}{2}
\subsection*{Contoh Masukan 1}
\begin{lstlisting}
Kasur
\end{lstlisting}
\columnbreak
\subsection*{Contoh Keluaran 1}
\begin{lstlisting}
rusak
\end{lstlisting}
\vfill
\null
\end{multicols}

\begin{multicols}{2}
\subsection*{Contoh Masukan 2}
\begin{lstlisting}
Ka$ur
\end{lstlisting}
\columnbreak
\subsection*{Contoh Keluaran 2}
\begin{lstlisting}
Emor tidak beruntung :(
\end{lstlisting}
\vfill
\null
\end{multicols}

\begin{multicols}{2}
\subsection*{Contoh Masukan 3}
\begin{lstlisting}
beRuntung123
\end{lstlisting}
\columnbreak
\subsection*{Contoh Keluaran 3}
\begin{lstlisting}
321gnutnureb
\end{lstlisting}
\vfill
\null
\end{multicols}

% \subsection*{Penjelasan}
\pagebreak

\end{document}