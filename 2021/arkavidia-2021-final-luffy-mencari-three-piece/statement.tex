\documentclass{article}

\usepackage{geometry}
\usepackage{amsmath}
\usepackage{graphicx, eso-pic}
\usepackage{listings}
\usepackage{hyperref}
\usepackage{multicol}
\usepackage{fancyhdr}
\pagestyle{fancy}
\fancyhf{}
\hypersetup{ colorlinks=true, linkcolor=black, filecolor=magenta, urlcolor=cyan}
\geometry{ a4paper, total={170mm,257mm}, top=40mm, right=20mm, bottom=20mm, left=20mm}
\setlength{\parindent}{0pt}
\setlength{\parskip}{0.3em}
\renewcommand{\headrulewidth}{0pt}
\AddToShipoutPictureBG{%
  \AtPageUpperLeft{%
    \raisebox{-\height}{\includegraphics[width=\paperwidth, height=30mm]{../headerarkav.png}}
  }
}
\rfoot{\thepage}
\lfoot{Final Competitive Programming - Arkavidia 7.0}
\lstset{
    basicstyle=\ttfamily\small,
    columns=fixed,
    extendedchars=true,
    breaklines=true,
    tabsize=2,
    prebreak=\raisebox{0ex}[0ex][0ex]{\ensuremath{\hookleftarrow}},
    frame=none,
    showtabs=false,
    showspaces=false,
    showstringspaces=false,
    prebreak={},
    keywordstyle=\color[rgb]{0.627,0.126,0.941},
    commentstyle=\color[rgb]{0.133,0.545,0.133},
    stringstyle=\color[rgb]{01,0,0},
    captionpos=t,
    escapeinside={(\%}{\%)}
}

\begin{document}

\begin{center}
    \section*{L - Luffy Mencari Three Pieces} % ganti judul soal

    \begin{tabular}{ | c c | }
        \hline
        Batas Waktu  & 2s \\    % jangan lupa ganti time limit
        Batas Memori & 256MB \\  % jangan lupa ganti memory limit
        \hline
    \end{tabular}
\end{center}

\subsection*{Deskripsi}

Luffy memiliki tiga buah bilangan asli $A_0$, $B_0$, $C_0$ dan $Q$ buah query.

Pada setiap query, diberikan $S$ dan $T$, dengan $S$ bernilai karakter \lstinline|a|, \lstinline|b|, atau \lstinline|c|, serta $T$ bilangan bulat. Untuk setiap query ke-$i$, apabila:

\begin{itemize}

\item $S =$ \lstinline|a|, maka $A_i = A_{i - 1} + T$, $B_i = B_{i - 1}$, $C_i = C_{i - 1}$

\item $S$ = \lstinline|b|, maka $A_i = A_{i - 1}$, $B_i = B_{i - 1} + T$, $C_i = C_{i - 1}$

\item $S$ = \lstinline|c|, maka $A_i = A_{i - 1}$, $B_i = B_{i - 1}$, $C_i = C_{i - 1} + T$

\end{itemize}

Kemudian untuk setiap query ke-$i$ tersebut, Luffy ingin mencari banyaknya tripel bilangan bulat non-negatif $\left(D, E, F\right)$ sehingga terdapat bilangan real $0 < R < 1$ yang memenuhi:
\begin{align*}
D &< R \times A_i < D + 1 \\
E &< R \times B_i < E + 1 \\
F &< R \times C_i < F + 1
\end{align*}

Sayangnya Luffy sedang sibuk berpetualang, dia kemudian meminta bantuan Anda untuk menyelesaikan persoalan ini. Bantulah Luffy!

\subsection*{Format Masukan}
Baris pertama berisi tiga buah bilangan $A_0, B_0, C_0$ $(0 < A_0, B_0, C_0 < 10^6)$.

Baris berikutnya berisi sebuah bilangan $Q$ $(1 \leq Q \leq 10^6)$.

$Q$ baris berikutnya berisi sebuah karakter $S$ (bernilai sebuah karakter diantara \lstinline|a|, \lstinline|b|, atau \lstinline|c|) dan sebuah bilangan bulat $T$ $(0 \leq T \leq 10^6)$ dipisah spasi.

\subsection*{Format Keluaran}
$Q$ baris yang berisi sebuah bilangan yang menyatakan banyaknya tripel bilangan bulat nonnegatif $\left(D, E, F\right)$.

\begin{multicols}{2}
\subsection*{Contoh Masukan}
\begin{lstlisting}
1 1 1
4
a 0
b 1
c 1
a 2
\end{lstlisting}
\columnbreak
\subsection*{Contoh Keluaran}
\begin{lstlisting}
1
2
2
4
\end{lstlisting}
\vfill
\null
\end{multicols}

\pagebreak
\subsection*{Penjelasan}
Penjelasannya adalah sebagai berikut:
\begin{itemize}

\item Setelah query pertama diperoleh $(A_1, B_1, C_1) = (1,1,1)$. Maka tripel $(D, E, F)$ yang memenuhi hanyalah $(0,0,0)$ yakni dapat diambil $R=0.5$, karena memenuhi $0<0.5<1$.

\item Setelah query kedua diperoleh $(A_2, B_2, C_2) = (1,2,1)$. Maka tripel $(D, E, F)$ yang memenuhi ialah $(0,0,0)$ dan $(0,1,0)$ yakni berturut-turut dapat diambil $R=0.2$ dan $R=0.8$.

\item Setelah query ketiga diperoleh $(A_3, B_3, C_3) = (1,2,2)$. Maka tripel $(D, E, F)$ yang memenuhi ialah $(0,0,0)$ dan $(0,1,1)$ yakni berturut-turut dapat diambil $R=0.2$ dan $R=0.8$.

\item Setelah query keempat diperoleh $(A_4, B_4, C_4) = (3,2,2)$. Maka tripel $(D, E, F)$ yang memenuhi ialah $(0,0,0)$, $(1,0,0)$, $(1,1,1)$ dan $(2,1,1)$ yakni berturut-turut dapat diambil $R=0.2$, $R=0.4$, $R=0.6$, dan $R=0.8$.

\end{itemize}


\end{document}