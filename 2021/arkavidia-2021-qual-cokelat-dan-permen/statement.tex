\documentclass{article}

\usepackage{geometry}
\usepackage{amsmath}
\usepackage{graphicx, eso-pic}
\usepackage{listings}
\usepackage{hyperref}
\usepackage{multicol}
\usepackage{fancyhdr}
\pagestyle{fancy}
\fancyhf{}
\hypersetup{ colorlinks=true, linkcolor=black, filecolor=magenta, urlcolor=cyan}
\geometry{ a4paper, total={170mm,257mm}, top=40mm, right=20mm, bottom=20mm, left=20mm}
\setlength{\parindent}{0pt}
\setlength{\parskip}{0.3em}
\renewcommand{\headrulewidth}{0pt}
\AddToShipoutPictureBG{%
  \AtPageUpperLeft{%
    \raisebox{-\height}{\includegraphics[width=\paperwidth, height=30mm]{../headerarkav.png}}
  }
}
\rfoot{\thepage}
\lfoot{Penyisihan Competitive Programming - Arkavidia 7.0}
\lstset{
    basicstyle=\ttfamily\small,
    columns=fixed,
    extendedchars=true,
    breaklines=true,
    tabsize=2,
    prebreak=\raisebox{0ex}[0ex][0ex]{\ensuremath{\hookleftarrow}},
    frame=none,
    showtabs=false,
    showspaces=false,
    showstringspaces=false,
    prebreak={},
    keywordstyle=\color[rgb]{0.627,0.126,0.941},
    commentstyle=\color[rgb]{0.133,0.545,0.133},
    stringstyle=\color[rgb]{01,0,0},
    captionpos=t,
    escapeinside={(\%}{\%)}
}

\begin{document}

\begin{center}
    \section*{C - Cokelat dan Permen} % ganti judul soal

    \begin{tabular}{ | c c | }
        \hline
        Batas Waktu  & 1s \\    % jangan lupa ganti time limit
        Batas Memori & 256MB \\  % jangan lupa ganti memory limit
        \hline
    \end{tabular}
\end{center}

\subsection*{Deskripsi}
Terdapat $N$ anak-anak yang memiliki cokelat dan permen. Setiap anak memiliki $A$ cokelat dan $B$ permen, anak ke-$i$ memiliki $A_i$ cokelat dan $B_i$ permen. Anda ingin memberikan beberapa cokelat dan permen kepada $N$ anak-anak tersebut sedemikian sehingga:

\begin{itemize}
\item Nilai dari $A + B$ sama untuk setiap anak, artinya semua anak memiliki jumlah total cokelat dan permen yang sama.
\item Untuk setiap $i, j$ $(1 \leq i, j \leq N, i \neq j)$ maka $A_i \neq A_j$ dan $B_i \neq B_j$, artinya untuk setiap dua anak berbeda maka jumlah permennya berbeda dan jumlah cokelatnya berbeda satu sama lain.
\end{itemize}

Berapakah total coklat dan permen minimal yang perlu disiapkan untuk memenuhi hal tersebut?
 	
\subsection*{Format Masukan}

Baris pertama berisi bilangan bulat $N$ $(1 \leq N \leq 10^5)$, menyatakan banyaknya anak.

Baris kedua berisi $N$ bilangan bulat $A_i$ $(1 \leq A_i \leq 10^{9})$, menyatakan banyaknya cokelat yang dimiliki anak ke-$i$ $(1 \leq i \leq N)$.

Baris ketiga berisi $N$ bilangan bulat $B_i$ $(1 \leq B_i \leq 10^{9})$, menyatakan banyaknya permen yang dimiliki anak ke-$i$ $(1 \leq i \leq N)$.

\subsection*{Format Keluaran}

Satu bilangan bulat berupa jawaban dari pertanyaan soal.

\begin{multicols}{2}
\subsection*{Contoh Masukan}
\begin{lstlisting}
4
6 2 2 3
5 5 8 6
\end{lstlisting}
\columnbreak
\subsection*{Contoh Keluaran}
\begin{lstlisting}
7
\end{lstlisting}
\vfill
\null
\end{multicols}

\subsection*{Penjelasan}

Dengan total 7 cokelat dan permen, atau lebih tepatnya 5 cokelat dan 2 permen, maka kita bisa memberikan cokelat dan permen sebagai berikut:
\begin{enumerate}
\item Anak pertama tidak perlu diberikan cokelat dan permen, sehingga $A_1=6$ dan $B_1=5$.
\item Anak kedua dapat diberikan 2 cokelat dan 2 permen, sehingga $A_2=4$ dan $B_2=7$.
\item Anak ketiga dapat diberikan 1 cokelat, sehingga $A_3=3$ dan $B_3=8$.
\item Anak keempat dapat diberikan 2 cokelat, sehingga $A_4=5$ dan $B_4=6$.
\end{enumerate}


\pagebreak

\end{document}