\documentclass{article}

\usepackage{geometry}
\usepackage{amsmath}
\usepackage{graphicx, eso-pic}
\usepackage{listings}
\usepackage{hyperref}
\usepackage{multicol}
\usepackage{fancyhdr}
\pagestyle{fancy}
\fancyhf{}
\hypersetup{ colorlinks=true, linkcolor=black, filecolor=magenta, urlcolor=cyan}
\geometry{ a4paper, total={170mm,257mm}, top=40mm, right=20mm, bottom=20mm, left=20mm}
\setlength{\parindent}{0pt}
\setlength{\parskip}{0.3em}
\renewcommand{\headrulewidth}{0pt}
\AddToShipoutPictureBG{%
  \AtPageUpperLeft{%
    \raisebox{-\height}{\includegraphics[width=\paperwidth, height=30mm]{../headerarkav.png}}
  }
}
\rfoot{\thepage}
\lfoot{Final Competitive Programming - Arkavidia 7.0}
\lstset{
    basicstyle=\ttfamily\small,
    columns=fixed,
    extendedchars=true,
    breaklines=true,
    tabsize=2,
    prebreak=\raisebox{0ex}[0ex][0ex]{\ensuremath{\hookleftarrow}},
    frame=none,
    showtabs=false,
    showspaces=false,
    showstringspaces=false,
    prebreak={},
    keywordstyle=\color[rgb]{0.627,0.126,0.941},
    commentstyle=\color[rgb]{0.133,0.545,0.133},
    stringstyle=\color[rgb]{01,0,0},
    captionpos=t,
    escapeinside={(\%}{\%)}
}

\begin{document}

\begin{center}
    \section*{F - Fungsi Acak} % ganti judul soal

    \begin{tabular}{ | c c | }
        \hline
        Batas Waktu  & 2s \\    % jangan lupa ganti time limit
        Batas Memori & 256MB \\  % jangan lupa ganti memory limit
        \hline
    \end{tabular}
\end{center}

\subsection*{Deskripsi}
Albert ingin membuat sebuah barisan yang terdiri atas $N$ buah bilangan $X_1, X_2, ..., X_N$. Ia diberikan $N$ buah interval, interval ke-$i$ adalah [$L_i, R_i$], dan bilangan ke-$i$ pada barisan tersebut harus memenuhi $L_i \leq X_i \leq R_i$.

Untuk setiap $i$, ketika menentukan nilai $X_i$, Albert menggunakan sebuah fungsi acak atau \textit{randomizer} pada interval ke-$i$ sehingga setiap bilangan pada interval tersebut dari $L_i$ hingga $R_i$ memiliki peluang yang sama untuk terpilih sebagai $X_i$, yaitu $\frac{1}{R_i-L_i+1}$. Albert akan sedih jika di antara $N$ bilangan tersebut banyak bilangan yang nilainya sama, sehingga ia ingin menghitung nilai rata-rata dari banyak bilangan berbeda pada barisan yang terpilih dari semua kemungkinan barisan yang dapat terpilih. Nilai rata-rata ini dapat ditulis dalam bentuk pecahan paling sederhana $P/Q$. Tentukan nilai dari $P \times Q^{-1} \pmod {10^{9}+7}$.

\subsection*{Format Masukan}
Baris pertama berisi sebuah bilangan bulat $N$ $(1 \leq N \leq 10^{5})$  yang menyatakan panjang barisan.

$N$ baris selanjutnya berisi pasangan bilangan bulat $L_i$ dan $R_i$ ($1 \leq L_i \leq R_i \leq 10^{5}$ untuk setiap $1 \leq i \leq N$), yang menyatakan interval ke-$i$ . 

\subsection*{Format Keluaran}
Keluaran terdiri dari satu baris, yaitu nilai dari $P \times Q^{-1} \mod (10^{9}+7)$, dengan $P$, $Q$ bilangan bulat dan $P/Q$ menyatakan nilai rata-rata dari banyak bilangan terpilih yang berbeda dari semua kemungkinan bilangan yang dihasilkan oleh \textit{randomizer} dalam bentuk pecahan paling sederhana.

\begin{multicols}{2}
\subsection*{Contoh Masukan 1}
\begin{lstlisting}
2
1 2
3 4
\end{lstlisting}
\columnbreak

\subsection*{Contoh Keluaran 1}
\begin{lstlisting}
2
\end{lstlisting}
\vfill
\null
\end{multicols}

\begin{multicols}{2}
\subsection*{Contoh Masukan 2}
\begin{lstlisting}
3
1 3
1 2
5 6
\end{lstlisting}
\columnbreak

\subsection*{Contoh Keluaran 2}
\begin{lstlisting}
666666674
\end{lstlisting}
\vfill
\null
\end{multicols}

\pagebreak
\subsection*{Penjelasan}

\begin{itemize}

\item Pada kasus uji pertama, terdapat empat kemungkinan barisan yang dapat dibuat oleh Albert:
$$\{1,3\}, \{1,4\}, \{2,3\}, \{2,4\}$$
Nilai rata-rata dari banyak bilangan yang berbeda pada masing-masing barisan tersebut adalah:
$$\frac{2+2+2+2}{4} = 2$$
\item Pada kasus uji kedua, terdapat 12 kemungkinan barisan yang dapat dibuat oleh Albert, yaitu:
    \begin{itemize}
        \item $\{1,1,5\}$, terdapat $2$ bilangan berbeda
        \item $\{1,1,6\}$, terdapat $2$ bilangan berbeda
        \item $\{1,2,5\}$, terdapat $3$ bilangan berbeda
        \item $\{1,2,6\}$, terdapat $3$ bilangan berbeda
        \item $\{2,1,5\}$, terdapat $3$ bilangan berbeda
        \item $\{2,1,6\}$, terdapat $3$ bilangan berbeda
        \item $\{2,2,5\}$, terdapat $2$ bilangan berbeda
        \item $\{2,2,6\}$, terdapat $2$ bilangan berbeda
        \item $\{3,1,5\}$, terdapat $3$ bilangan berbeda
        \item $\{3,1,6\}$, terdapat $3$ bilangan berbeda
        \item $\{3,2,5\}$, terdapat $3$ bilangan berbeda
        \item $\{3,2,6\}$, terdapat $3$ bilangan berbeda
    \end{itemize}
    Nilai rata-rata dari banyak bilangan berbeda pada masing-masing barisan tersebut adalah:
    $$\frac{2+2+3+3+3+3+2+2+3+3+3+3}{12} = \frac{32}{12} = \frac{8}{3},$$
    dengan
    $$ 8 \cdot 3^{-1} \equiv 666666674 \pmod {10^9 + 7}.$$

\end{itemize} 

\end{document}