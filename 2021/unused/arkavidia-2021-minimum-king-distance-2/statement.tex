\documentclass{article}

\usepackage{geometry}
\usepackage{amsmath}
\usepackage{graphicx}
\usepackage{listings}
\usepackage{hyperref}
\usepackage{multicol}
\usepackage{fancyhdr}
\pagestyle{fancy}
\hypersetup{ colorlinks=true, linkcolor=black, filecolor=magenta, urlcolor=cyan}
\geometry{ a4paper, total={170mm,257mm}, top=20mm, right=20mm, bottom=20mm, left=20mm}
\setlength{\parindent}{0pt}
\setlength{\parskip}{0.3em}
\renewcommand{\headrulewidth}{0pt}
\lhead{Penyisihan Competitive Programming - Arkavidia VII}
\fancyfoot[CE,CO]{\thepage}
\lstset{
    basicstyle=\ttfamily\small,
    columns=fixed,
    extendedchars=true,
    breaklines=true,
    tabsize=2,
    prebreak=\raisebox{0ex}[0ex][0ex]{\ensuremath{\hookleftarrow}},
    frame=none,
    showtabs=false,
    showspaces=false,
    showstringspaces=false,
    prebreak={},
    keywordstyle=\color[rgb]{0.627,0.126,0.941},
    commentstyle=\color[rgb]{0.133,0.545,0.133},
    stringstyle=\color[rgb]{01,0,0},
    captionpos=t,
    escapeinside={(\%}{\%)}
}

\begin{document}

\begin{center}
    \section*{Minimum King Distance 2} % ganti judul soal

    \begin{tabular}{ | c c | }
        \hline
        Batas Waktu  & 1s \\    % jangan lupa ganti time limit
        Batas Memori & 256MB \\  % jangan lupa ganti memory limit
        \hline
    \end{tabular}
\end{center}

\subsection*{Deskripsi}
Nonki mendapat tantangan dari Locel! Locel memberikan Nonki sebuah papan catur berukuran $M \times M$, dua buah array \textit{one-indexed} berukuran $N$, sebut saja array $R$ dan $C$, dan sebuah \textit{piece} \textbf{raja}. Lalu, Locel meminta Nonki melakukan hal berikut:

\begin{enumerate}
\item \textit{Rearrange} (menyusun ulang) array $C$ sesuka hati.
\item Pilih dua buah integer berbeda $i$ dan $j$ dalam \textit{range} $[1..N]$.
\item Letakkan piece raja pada sel $(R_i,C_i)$, lalu jalankan ke sel $(R_j,C_j)$ dengan aturan gerakan raja. Jalur yang dilalui piece raja bebas. Sel $(R_i,C_i)$ dan $(R_j,C_j)$ boleh sama.
\end{enumerate}

Misalkan banyak langkah yang dilakukan \textit{piece} raja adalah $D$. Locel menantang Nonki untuk mendapatkan nilai $D$ sekecil mungkin. Carilah nilai $D$ \textbf{terkecil} yang bisa didapatkan Nonki!

\subsection*{Format Masukan}
Baris pertama terdiri dari dua buah integer $M$ $(1 \leq M \leq 10^{15})$ dan $N$ $(1 \leq N \leq 10^5)$, masing-masing menyatakan ukuran sisi papan catur dan panjang array. 

Baris kedua terdiri dari $N$ buah integer $R_1,R_2,...,R_N$ $(1 \leq R_i \leq M)$ yang menyatakan elemen array $R$.

Baris ketiga terdiri dari $N$ buah integer $C_1,C_2,...,C_N$ $(1 \leq C_i \leq M)$ yang menyatakan elemen array $C$.

\subsection*{Format Keluaran}
Keluaran berupa integer $D_{min}$, yang menyatakan banyak langkah minimum yang dapat dilakukan \textit{piece} raja.

\begin{multicols}{2}
\subsection*{Contoh Masukan}
\begin{lstlisting}
10 5
9 1 6 7 2
4 10 4 8 3

\end{lstlisting}
\columnbreak
\subsection*{Contoh Keluaran}
\begin{lstlisting}
1
\end{lstlisting}
\vfill
\null
\end{multicols}

\subsection*{Penjelasan}
Nonki dapat menyusun ulang array $C$ menjadi $[8, 10, 4, 4, 3]$.
\\\\
Lalu, Nonki dapat memilih $i=3$ dan $j=4$, meletakkan \textit{piece} raja pada sel (6,4), dan menjalankan \textit{piece} raja ke sel (7,4) dalam 1 langkah.

\end{document}