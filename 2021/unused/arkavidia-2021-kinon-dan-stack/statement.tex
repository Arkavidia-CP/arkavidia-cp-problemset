\documentclass{article}

\usepackage{geometry}
\usepackage{amsmath}
\usepackage{graphicx}
\usepackage{listings}
\usepackage{hyperref}
\usepackage{multicol}
\usepackage{fancyhdr}
\pagestyle{fancy}
\hypersetup{ colorlinks=true, linkcolor=black, filecolor=magenta, urlcolor=cyan}
\geometry{ a4paper, total={170mm,257mm}, top=20mm, right=20mm, bottom=20mm, left=20mm}
\setlength{\parindent}{0pt}
\setlength{\parskip}{0.3em}
\renewcommand{\headrulewidth}{0pt}
\lhead{Penyisihan Competitive Programming - Arkavidia VII}
\fancyfoot[CE,CO]{\thepage}
\lstset{
    basicstyle=\ttfamily\small,
    columns=fixed,
    extendedchars=true,
    breaklines=true,
    tabsize=2,
    prebreak=\raisebox{0ex}[0ex][0ex]{\ensuremath{\hookleftarrow}},
    frame=none,
    showtabs=false,
    showspaces=false,
    showstringspaces=false,
    prebreak={},
    keywordstyle=\color[rgb]{0.627,0.126,0.941},
    commentstyle=\color[rgb]{0.133,0.545,0.133},
    stringstyle=\color[rgb]{01,0,0},
    captionpos=t,
    escapeinside={(\%}{\%)}
}

\begin{document}

\begin{center}
    \section*{Kinon dan Stack} % ganti judul soal

    \begin{tabular}{ | c c | }
        \hline
        Batas Waktu  & 2s \\    % jangan lupa ganti time limit
        Batas Memori & 256MB \\  % jangan lupa ganti memory limit
        \hline
    \end{tabular}
\end{center}

\subsection*{Deskripsi}
Kinon mempunyai $M$ buah stack. Kinon telah melakukan $Q$ operasi dengan stack-stack tersebut. Operasi ke-$i$ berbentuk integer $X_i$.
\begin{enumerate}
\item Apabila $X_i>0$, Kinon meletakkan elemen $X_i$ pada puncak salah satu stack
\item Apabila $X_i<0$, Kinon mengambil elemen $-X_i$ dari puncak salah satu stack.
\end{enumerate}

Tepat sebelum Kinon melakukan operasi jenis ke-2, pasti terdapat elemen $-X_i$ pada puncak salah satu stack. Selain itu, stack-stack Kinon mungkin tidak kosong setelah $Q$ operasi tersebut.
Kinon yakin bahwa $Q$ buah operasinya tidak dapat dilakukan apabila jumlah stacknya kurang dari $M$. Bantulah Kinon untuk menemukan nilai $M$!

\subsection*{Format Masukan}
Baris pertama berisi bilangan bulat $Q$ $(1 \leq Q \leq 10^5)$, menyatakan banyaknya operasi.

Baris berikutnya berisi $Q$ bilangan bulat, $X_i$ $(1 \leq X_i \leq 10^5)$, menyatakan bilangan pada operasi ke-$i$.

\subsection*{Format Keluaran}
Satu baris berisi nilai $M$ sesuai dengan deskripsi soal.

\begin{multicols}{2}
\subsection*{Contoh Masukan 1}
\begin{lstlisting}
6
1 3 5 7 -3 -1
\end{lstlisting}
\columnbreak
\subsection*{Contoh Keluaran 1}
\begin{lstlisting}
2
\end{lstlisting}
\vfill
\null
\end{multicols}

\begin{multicols}{2}
\subsection*{Contoh Masukan 2}
\begin{lstlisting}
8
1 2 3 -2 -1 1 -3 2
\end{lstlisting}
\columnbreak
\subsection*{Contoh Keluaran 2}
\begin{lstlisting}
2
\end{lstlisting}
\vfill
\null
\end{multicols}


\subsection*{Penjelasan 1}
Berikut salah satu kemungkinan isi masing-masing stack Kinon selama operasi:

\begin{enumerate}

\item{[1], [ ]}
\item{[1,3], [ ]}
\item{[1,3], [5]}
\item{[1,3], [5,7]}
\item{[1], [5,7]}
\item{[ ], [5,7]}

\end{enumerate}

Terlihat bahwa 6 operasi tersebut dapat dilakukan dengan $M=2$ stack. Dapat ditunjukkan bahwa 6 operasi tersebut tidak dapat dilakukan dengan $M<2$ stack. 

\subsection*{Penjelasan 2}
Berikut salah satu kemungkinan isi masing-masing stack Kinon selama operasi:

\begin{enumerate}

\item{[1], [ ]}
\item{[1,2], [ ]}
\item{[1,2], [3]}
\item{[1], [3]}
\item{[ ], [3]}
\item{[1], [3]}
\item{[1], [ ]}
\item{[1, 2], [ ]}

\end{enumerate}

Terlihat bahwa 8 operasi tersebut dapat dilakukan dengan $M=2$ stack. Dapat ditunjukkan bahwa 8 operasi tersebut tidak dapat dilakukan dengan $M<2$ stack.
\end{document}