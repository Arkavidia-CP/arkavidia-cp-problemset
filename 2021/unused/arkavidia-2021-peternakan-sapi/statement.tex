\documentclass{article}

\usepackage{geometry}
\usepackage{amsmath}
\usepackage{graphicx}
\usepackage{listings}
\usepackage{hyperref}
\usepackage{multicol}
\usepackage{fancyhdr}
\pagestyle{fancy}
\hypersetup{ colorlinks=true, linkcolor=black, filecolor=magenta, urlcolor=cyan}
\geometry{ a4paper, total={170mm,257mm}, top=20mm, right=20mm, bottom=20mm, left=20mm}
\setlength{\parindent}{0pt}
\setlength{\parskip}{0.3em}
\renewcommand{\headrulewidth}{0pt}
\lhead{Penyisihan Competitive Programming - Arkavidia VII}
\fancyfoot[CE,CO]{\thepage}
\lstset{
    basicstyle=\ttfamily\small,
    columns=fixed,
    extendedchars=true,
    breaklines=true,
    tabsize=2,
    prebreak=\raisebox{0ex}[0ex][0ex]{\ensuremath{\hookleftarrow}},
    frame=none,
    showtabs=false,
    showspaces=false,
    showstringspaces=false,
    prebreak={},
    keywordstyle=\color[rgb]{0.627,0.126,0.941},
    commentstyle=\color[rgb]{0.133,0.545,0.133},
    stringstyle=\color[rgb]{01,0,0},
    captionpos=t,
    escapeinside={(\%}{\%)}
}

\begin{document}

\begin{center}
    \section*{Peternakan Sapi} % ganti judul soal

    \begin{tabular}{ | c c | }
        \hline
        Batas Waktu  & 1s \\    % jangan lupa ganti time limit
        Batas Memori & 128MB \\  % jangan lupa ganti memory limit
        \hline
    \end{tabular}
\end{center}

\subsection*{Deskripsi}

Pada Peternakan Arkavidia, terdapat $N$ sapi dan $N$ slot, masing-masing dari sapi tersebut berada di satu slot dan slot tersebut dinomori dari $1$ sampai $N$. Pada awalnya, sapi ke-$i$ berada pada slot ke-$i$. Setiap sapi ke-$i$ memiliki tingkat kehausan $H_i$, dan dapat berubah tingkat kehausannya sewaktu-waktu. Arvy, seorang peternak disana akan mengurusi beberapa hal pada peternakan tersebut:

\begin{enumerate}
\item Memberikan minuman sapi-sapi yang berada pada slot ke-$L$ sampai slot ke-$R$ ($L \leq R$), untuk hal ini, Arvy perlu mengetahui nilai tingkat kehausan minimal dan maksimal sapi-sapi dari $L$ sampai $R$ (inklusif) agar bisa memberikan minuman yang optimal.
\item Menukar posisi sapi pada slot ke-$X$ dengan posisi sapi pada slot ke-$Y$ $(X \neq Y)$.
\item Memberikan minuman khusus untuk sapi ke-$X$, yang dapat mengubah tingkat kehausannya menjadi $V$.
\item Melakukan \textit{reverse} posisi sapi dari slot ke-$L$ sampai slot ke-$R$, artinya:

\begin{center}
Posisi yang awalnya $\left[L, L + 1, \dots, R - 1, R\right]$ menjadi $\left[R, R - 1, \dots, L + 1, L\right]$
\end{center}
\end{enumerate}

Setelah peternakan-nya semakin membesar, Arvy mulai kesusahan dan meminta bantuan Anda untuk mengurusi peternakan. Bantulah Arvy!

\subsection*{Format Masukan}
Baris pertama berisi bilangan bulat positif $N$, $(2 \leq N \leq 10^5)$, menyatakan banyaknya sapi yang ada.

Baris kedua berisi $N$ bilangan bulat positif, $H_i$ $(1 \leq H_i \leq 10^9)$, untuk setiap $i$ $(1 \leq i \leq N)$, menyatakan tingkat kehausan sapi ke-$i$.

Baris ketiga berisi bilangan bulat positif $Q$, $(1 \leq Q \leq 10^5)$, menyatakan operasi yang perlu dilakukan Arvy.

Setiap $Q$ baris selanjutnya berisi salah satu dari tiga bentuk masukkan berikut:
\begin{itemize}
\item $1$ $L$ $R$ $(1 \leq L \leq R \leq N)$
(Mencari nilai minimum dan maksimum tingkat kehausan sapi dari slot ke-$L$ sampai slot ke-$R$)
\item $2$ $X$ $Y$ $(1 \leq X, Y \leq N, X \neq Y)$
(Menukar posisi sapi di slot ke-$X$ dengan posisi sapi di slot ke-$Y$)
\item $3$ $X$ $V$ $(1 \leq X \leq N, 1 \leq V \leq 10^9)$
(Mengubah tingkat kehausan sapi di slot ke-$X$ dengan nilai $V$)
\item $4$ $L$ $R$ $(1 \leq L \leq R \leq N)$
(Melakukan reverse posisi sapi dari slot ke-$L$ sampai slot ke-$R$)
\end{itemize}

\textbf{Catatan:}

Dijamin terdapat minimal satu buah operasi bentuk pertama

\subsection*{Format Keluaran}
Untuk setiap operasi bentuk pertama keluarkan sebuah baris berisi dua bilangan bulat positif, menyatakan nilai minimum dan nilai maksimum tingkat kehausan sapi yang dicari.

\pagebreak

\begin{multicols}{2}
\subsection*{Contoh Masukan}
\begin{lstlisting}
5
1 5 2 3 9
10
1 3 5
3 3 10
1 3 5
1 1 5
1 1 3
2 1 5
4 1 3
1 2 5
3 3 6
1 2 4

\end{lstlisting}
\columnbreak
\subsection*{Contoh Keluaran}
\begin{lstlisting}
2 9
3 10
1 10
1 10
1 9
3 6
\end{lstlisting}
\vfill
\null
\end{multicols}

\subsection*{Penjelasan}
Pada awalnya, tingkat kehausan setiap sapi dari slot 1 sampai 5 adalah [1, 5, 2, 3, 9]
\begin{itemize}
\item Pada operasi ke-1, nilai minimum dan maksimum tingkat kehausan dari sapi di slot-3 sampai slot-5 adalah 2 dan 9
\item Pada operasi ke-2, sapi di slot-3 berubah tingkat kehausannya menjadi 10, sehingga tingkat kehausan masing-masing sapi di tiap slot menjadi [1, 5, 10, 3, 9]
\item Pada operasi ke-3, nilai minimum dan maksimum tingkat kehausan dari sapi di slot-3 sampai slot-5 adalah 3 dan 10
\item Pada operasi ke-4, nilai minimum dan maksimum tingkat kehausan dari sapi di slot-1 sampai slot-5 adalah 1 dan 10
\item Pada operasi ke-5, nilai minimum dan maksimum tingkat kehausan dari sapi di slot-1 sampai slot-3 adalah 1 dan 10
\item Pada operasi ke-6, sapi di slot-1 ditukar dengan sapi di slot-5, sehingga tingkat kehausan masing-masing sapi di tiap slot menjadi [9, 5, 10, 3, 1]
\item Pada operasi ke-7, lakukan reverse slot-1 sampai slot-3, sehingga tingkat kehausan masing-masing sapi di tiap slot menjadi [10, 5, 9, 3, 1]
\item Pada operasi ke-8, nilai minimum dan maksimum tingkat kehausan dari sapi di slot-2 sampai slot-5 adalah 1 dan 9
\item Pada operasi ke-9, sapi di slot-3 berubah tingkat kehausannya menjadi 6, sehingga tingkat kehausan masing-masing sapi di tiap slot menjadi [10, 5, 6, 3, 1]
\item Pada operasi ke-10, nilai minimum dan maksimum tingkat kehausan dari sapi di slot-2 sampai slot-5 adalah 3 dan 6
\end{itemize}

\end{document}
