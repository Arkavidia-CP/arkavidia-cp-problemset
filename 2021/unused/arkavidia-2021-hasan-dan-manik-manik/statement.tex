\documentclass{article}

\usepackage{geometry}
\usepackage{amsmath}
\usepackage{graphicx}
\usepackage{listings}
\usepackage{hyperref}
\usepackage{multicol}
\usepackage{fancyhdr}
\pagestyle{fancy}
\hypersetup{ colorlinks=true, linkcolor=black, filecolor=magenta, urlcolor=cyan}
\geometry{ a4paper, total={170mm,257mm}, top=20mm, right=20mm, bottom=20mm, left=20mm}
\setlength{\parindent}{0pt}
\setlength{\parskip}{0.3em}
\renewcommand{\headrulewidth}{0pt}
\lhead{Penyisihan Competitive Programming - Arkavidia VII}
\fancyfoot[CE,CO]{\thepage}
\lstset{
    basicstyle=\ttfamily\small,
    columns=fixed,
    extendedchars=true,
    breaklines=true,
    tabsize=2,
    prebreak=\raisebox{0ex}[0ex][0ex]{\ensuremath{\hookleftarrow}},
    frame=none,
    showtabs=false,
    showspaces=false,
    showstringspaces=false,
    prebreak={},
    keywordstyle=\color[rgb]{0.627,0.126,0.941},
    commentstyle=\color[rgb]{0.133,0.545,0.133},
    stringstyle=\color[rgb]{01,0,0},
    captionpos=t,
    escapeinside={(\%}{\%)}
}

\begin{document}

\begin{center}
    \section*{Hasan dan Manik-Manik} % ganti judul soal

    \begin{tabular}{ | c c | }
        \hline
        Batas Waktu  & 4s \\    % jangan lupa ganti time limit
        Batas Memori & 512MB \\  % jangan lupa ganti memory limit
        \hline
    \end{tabular}
\end{center}

\subsection*{Deskripsi}

Hasan mempunyai manik-manik yang disusun menjadi bentuk segitiga sama sisi dengan ketinggian $N$. 
Segitiga tersebut memiliki $1$ manik-manik di baris pertama, $2$ manik-manik di baris kedua, hingga 
$N$ manik-manik di baris ke-$N$. Hasan ingin mewarnai manik-manik tersebut dengan $K$ warna yang berbeda. 
Cara Hasan mewarnai manik-manik tersebut adalah sebagai berikut: 
\begin{itemize}
\item Warnai semua manik-manik pada \textit{medial triangle}-nya dengan warna ke-$i$. \textit{Medial triangle} 
adalah segitiga sama sisi yang titik-titik sudutnya berada di tengah dari setiap sisi segitiga terluar. Segitiga dengan ketinggian genap tidak memiliki \textit{medial triangle}.

\item Pada langkah ke-$i$, pilih sebuah subsegitiga sama sisi dengan ketinggian ganjik pada manik-manik tersebut 
yang belum diwarnai dan bukan merupakan bagian dari segitiga sama sisi lain yang belum diwarnai. Jika tidak ada, 
maka proses pewarnaan dihentikan.
\end{itemize}
Proses pewarnaan dilakukan maksimal sebanyak $K$ langkah, atau sebanyak warna yang Hasan miliki. Tidak menggunakan 
seluruh warna atau tidak mewarnai sama sekali juga terhitung sebagai salah satu cara pewarnaan. Dapatkah kamu 
menentukan berapa cara pewarnaan berbeda yang dapat dilakukan Hasan?

\subsection*{Format Masukan}
Masukkan berupa satu baris yang berisi $N$ $(1 \leq N \leq 10^{18})$ dan $K$ $(1 \leq K \leq 10^5)$ dipisah dengan spasi.

\subsection*{Format Keluaran}
Keluarkan banyak cara perwarnaan yang dapat dilakukan Hasan. 

\begin{multicols}{2}
\subsection*{Contoh Masukan}
\begin{lstlisting}
3 2
\end{lstlisting}
\columnbreak

\subsection*{Contoh Keluaran}
\begin{lstlisting}
5
\end{lstlisting}
\vfill
\null
\end{multicols}

\subsection*{Penjelasan}
Kelima cara tersebut adalah:
\begin{enumerate}
    \item Tidak mewarnai sama sekali
    \item Mewarnai \textit{medial triangle} dengan warna pertama
    \item Mewarnai \textit{medial triangle} dengan warna pertama dan subsegitiga dengan tinggi $1$ di atas \textit{medial triangle} dengan warna kedua
    \item Mewarnai \textit{medial triangle} dengan warna pertama dan subsegitiga dengan tinggi $1$ di kiri \textit{medial triangle} dengan warna kedua
    \item Mewarnai \textit{medial triangle} dengan warna pertama dan subsegitiga dengan tinggi $1$ di kanan \textit{medial triangle} dengan warna kedua
\end{enumerate}
\pagebreak

\end{document}