\documentclass{article}

\usepackage{geometry}
\usepackage{amsmath}
\usepackage{graphicx, eso-pic}
\usepackage{listings}
\usepackage{hyperref}
\usepackage{multicol}
\usepackage{fancyhdr}
\pagestyle{fancy}
\fancyhf{}
\hypersetup{ colorlinks=true, linkcolor=black, filecolor=magenta, urlcolor=cyan}
\geometry{ a4paper, total={170mm,257mm}, top=40mm, right=20mm, bottom=20mm, left=20mm}
\setlength{\parindent}{0pt}
\setlength{\parskip}{0.3em}
\renewcommand{\headrulewidth}{0pt}
\AddToShipoutPictureBG{%
  \AtPageUpperLeft{%
    \raisebox{-\height}{\includegraphics[width=\paperwidth, height=30mm]{../headerarkav.png}}
  }
}
\rfoot{\thepage}
\lfoot{Penyisihan Competitive Programming - Arkavidia 7.0}
\lstset{
    basicstyle=\ttfamily\small,
    columns=fixed,
    extendedchars=true,
    breaklines=true,
    tabsize=2,
    prebreak=\raisebox{0ex}[0ex][0ex]{\ensuremath{\hookleftarrow}},
    frame=none,
    showtabs=false,
    showspaces=false,
    showstringspaces=false,
    prebreak={},
    keywordstyle=\color[rgb]{0.627,0.126,0.941},
    commentstyle=\color[rgb]{0.133,0.545,0.133},
    stringstyle=\color[rgb]{01,0,0},
    captionpos=t,
    escapeinside={(\%}{\%)}
}

\begin{document}

\begin{center}
    \section*{I - Indra dan Pizza} % ganti judul soal

    \begin{tabular}{ | c c | }
        \hline
        Batas Waktu  & 1s \\    % jangan lupa ganti time limit
        Batas Memori & 256MB \\  % jangan lupa ganti memory limit
        \hline
    \end{tabular}
\end{center}

\subsection*{Deskripsi}

Pada Restoran Pizza Arkavidia, terdapat $N$ piring dan $M$ tumpukan piring. Setiap $N$ piring tersebut dinomori dari $1$ sampai $N$. Pada awalnya, beberapa dari $M$ tumpukan piring tersebut sudah ditumpuk dengan beberapa piring.

\begin{center}

\includegraphics[scale=0.5]{pizza.png}

\end{center}

Terdapat tiga operasi yang perlu dilakukan oleh Indra, seorang pengurus piring di restoran Arkavidia. Tiga operasi tersebut adalah sebagai berikut:

\begin{enumerate}
\item Meletakkan piring dengan nomor $P$ menjadi piring teratas dari tumpukan $T$
\item Mengambil piring teratas pada tumpukan $T$ dan memberikannya kepada pelayan restoran.
\item Memindahkan seluruh piring pada tumpukan $T_1$ ke atas tumpukan $T_2$. Setelah operasi ini, tumpukan $T_1$ akan menjadi kosong.
\end{enumerate}

Setelah melakukan semua operasi tersebut, kini Indra meminta bantuan Anda, untuk memberi tahu pada setiap $M$ tumpukan piring, terdapat piring apa saja yang ada di tumpukkan tersebut secara berurutan. Bantulah Indra!

\subsection*{Format Masukan}
Baris pertama berisi tiga bilangan bulat positif $N$, $M$, dan $Q$ $(2 \leq N, M, Q \leq 10^5, M \leq N)$ masing-masing menyatakan banyaknya piring, banyaknya tumpukan, dan banyaknya operasi.

$M$ baris berikutnya mendeskripsikan posisi piring pada setiap $M$ tumpukkan.

Setiap baris ke-$i$ dengan $1 \leq i \leq M$ diawali dengan bilangan bulat $L_i$ $(0 \leq L_i \leq N)$, menyatakan banyaknya piring yang ada pada tumpukan ke-$i$ tersebut. Kemudian dilanjutkan dengan $L_i$ angka yang menyatakan nomor dari piring pada tumpukkan tersebut, mengurut dari piring terbawah hingga piring teratas.

$Q$ baris berikutnya mendeskripsikan operasi-operasi yang akan dilakukan oleh Indra.

Setiap $Q$ baris tersebut merupakan salah satu dari tiga operasi yang ada:
\begin{enumerate}
\item $1$ $P$ $T$ $(1 \leq P \leq N)$ $(1 \leq T \leq M)$

(Meletakkan piring dengan nomor $P$ menjadi piring teratas dari tumpukan $T$)
\item $2$ $T$ $(1 \leq T \leq M)$

(Mengambil piring teratas pada tumpukan $T$)
\item $3$ $T_1$ $T_2$ $(1 \leq T_1, T_2 \leq M, T_1 \neq T_2)$

(Memindahkan seluruh piring pada tumpukan $T_1$ ke atas tumpukan $T_2$)
\end{enumerate}

\pagebreak
\textbf{Catatan:}
\begin{itemize}
\item Nomor setiap piring dijamin unik setiap saat, dan berada pada \textit{range} $[1, N]$.
\item Semua jumlah piring pada tumpukkan kurang dari atau sama dengan $N$ $(\sum_{i=1}^{M} L_i \leq N)$.
\item Pada operasi ke-1, piring bernomor $P$ dijamin belum ada pada tumpukan manapun.
\item Pada operasi ke-2, tumpukkan $T$ dijamin memiliki minimal satu piring.
\item Pada operasi ke-3, tumpukkan $T_1$ atau $T_2$ mungkin kosong.
\end{itemize}

\subsection*{Format Keluaran}
Untuk \textbf{setiap tumpukkan yang tidak kosong} dari $1$ sampai $M$, keluarkan nomor tumpukkan piring, diikuti dengan ": " (tanpa tanda petik), dan diikuti dengan nomor piring mulai dari terbawah sampai teratas yang ada pada tumpukkan tersebut dipisahkan dengan spasi.

\begin{multicols}{2}
\subsection*{Contoh Masukan}
\begin{lstlisting}
10 5 5
3 3 1 5
2 2 4
3 6 7 9
1 8
1 10
2 4
2 5
3 3 1
1 8 5
1 10 5

\end{lstlisting}
\columnbreak
\subsection*{Contoh Keluaran}
\begin{lstlisting}
1: 3 1 5 6 7 9
2: 2 4
5: 8 10
\end{lstlisting}
\vfill
\null
\end{multicols}

\subsection*{Penjelasan}
Pada awalnya, untuk setiap tumpukkan dari 1 sampai 5 kita mempunyai isi piring sebagai berikut:
\begin{enumerate}
\item $[$3, 1, 5$]$
\item $[$2, 4$]$
\item $[$6, 7, 9$]$
\item $[$8$]$
\item $[$10$]$
\end{enumerate}
Setelah Query-1, mengambil piring teratas dari tumpukan ke-4, tumpukkan menjadi:
\begin{enumerate}
\item $[$3, 1, 5$]$
\item $[$2, 4$]$
\item $[$6, 7, 9$]$
\item $[]$
\item $[$10$]$
\end{enumerate}
Setelah Query-2, mengambil piring teratas dari tumpukan ke-5, tumpukkan menjadi:
\begin{enumerate}
\item $[$3, 1, 5$]$
\item $[$2, 4$]$
\item $[$6, 7, 9$]$
\item $[]$
\item $[]$
\end{enumerate}
Setelah Query-3, memindahkan piring-piring pada tumpukan ke-3 ke tumpukan ke-1, tumpukkan menjadi:
\begin{enumerate}
\item $[$3, 1, 5, 6, 7, 9$]$
\item $[$2, 4$]$
\item $[]$
\item $[]$
\item $[]$
\end{enumerate}
Setelah Query-4, meletakkan piring nomor 8 ke tumpukan ke-5, tumpukkan menjadi:
\begin{enumerate}
\item $[$3, 1, 5, 6, 7, 9$]$
\item $[$2, 4$]$
\item $[]$
\item $[]$
\item $[8]$
\end{enumerate}
Setelah Query-5, meletakkan piring nomor 10 ke tumpukan ke-5, tumpukkan menjadi:
\begin{enumerate}
\item $[$3, 1, 5, 6, 7, 9$]$
\item $[$2, 4$]$
\item $[]$
\item $[]$
\item $[8, 10]$
\end{enumerate}

\end{document}