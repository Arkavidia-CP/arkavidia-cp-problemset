\section*{Problem I - Indra dan Pizza}
\addcontentsline{toc}{section}{Problem I - Indra dan Pizza}
\textit{Author: Ryo Richardo}
\\
\textit{Expected Difficulty: Easy-Medium}

Solusi dari permasalahan ini dapat diselesaikan menggunakan \textbf{linked-list}. Pada solusi ini, mula-mula definisikan tipe bentukan linked list yang mengandung \lstinline|array el| untuk menyimpan \lstinline|id| setiap piring pizza, \lstinline|count| untuk menyimpan jumlah piring dalam tumpukan, \lstinline|id| untuk menyimpan urutan tumpukan dari tumpukan pertama, dan \lstinline|next| untuk menyimpan alamat tumpukan selanjutnya.

Kemudian buatlah fungsi dan prosedur yang dapat mengkoordinasikan piring-piring pizza ke dalam tumpukan sesuai permintaan, yaitu fungsi \lstinline|Alokasi| (membentuk tumpukan baru), fungsi \lstinline|Search| (mencari tumpukan dengan \lstinline|id| tertentu), prosedur \lstinline|Insert| (memanggil fungsi \lstinline|Alokasi|, kemudian memasukan tumpukan yang terbentuk sebagai elemen baru linked-list), dan prosedur \lstinline|Enqueue| (memasukan piring dengan \lstinline|id| tertentu ke tumpukan dengan \lstinline|id| tertentu; jika \lstinline|id| tumpukan tidak ditemukan, maka memanggil prosedur \lstinline|Insert|)

Pada program utama, mula-mula masukkan nilai $N$, $M$, dan $Q$ sesuai spesifikasi soal. Kemudian, untuk menerima kondisi tumpukan awal, digunakan prosedur \lstinline|Insert| (tumpukan ke linked-list) dan \lstinline|Enqueue| (piring ke tumpukan). Kemudian, untuk query jenis 1, gunakan \lstinline|Enqueue| untuk memasukkan piring ke tumpukan. Untuk query jenis 2, gunakan fungsi \lstinline|Search| untuk mencari tumpukan dengan index tertentu, yang kemudian dikurangi elemen teratasnya. Untuk query jenis 3, gunakan fungsi \lstinline|Search| untuk mencari kedua \lstinline|id| tumpukan, serta prosedur \lstinline|Enqueue| yang diiterasikan dari tumpukan satu ke tumpukan yang lainnya.

Setelah selesai menjalankan semua query, program menampilkan tumpukan yang memiliki minimal satu buah piring. Caranya adalah dengan mengiterasikan semua elemen pada linked-list, melakukan filter untuk tumpukan dengan 0 piring, kemudian melakukan iterasi untuk setiap piring yang ada pada tumpukan tersebut.

Kita juga dapat menggunakan STL List pada C++ yang dapat dilihat di dokumentasi C++. Gunakan \lstinline|push_back| untuk query jenis 1, \lstinline|pop_back| untuk query jenis 2, dan splice untuk query jenis 3.

Kode Solusi: \url{https://ideone.com/eyWE8o}

Kompleksitas Waktu: $O(N)$

Catatan Panitia:

\begin{itemize}
    \item Awalnya soal ini diniatkan untuk diselesaikan dengan bruteforce, ternyata bisa digunakan linked list, sehingga soal dimodifikasi lagi
    \item Saat lomba dilaksanakan kami baru tau STL List pada C++ bisa digunakan untuk menyelesaikan soal ini.
\end{itemize}