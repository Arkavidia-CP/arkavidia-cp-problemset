\documentclass{article}

\usepackage{geometry}
\usepackage{amsmath}
\usepackage{graphicx, eso-pic}
\usepackage{listings}
\usepackage{hyperref}
\usepackage{multicol}
\usepackage{fancyhdr}
\pagestyle{fancy}
\fancyhf{}
\hypersetup{ colorlinks=true, linkcolor=black, filecolor=magenta, urlcolor=cyan}
\geometry{ a4paper, total={170mm,257mm}, top=40mm, right=20mm, bottom=20mm, left=20mm}
\setlength{\parindent}{0pt}
\setlength{\parskip}{0.3em}
\renewcommand{\headrulewidth}{0pt}
\AddToShipoutPictureBG{%
  \AtPageUpperLeft{%
    \raisebox{-\height}{\includegraphics[width=\paperwidth, height=30mm]{../headerarkav.png}}
  }
}
\rfoot{\thepage}
\lfoot{Final Competitive Programming - Arkavidia 7.0}
\lstset{
    basicstyle=\ttfamily\small,
    columns=fixed,
    extendedchars=true,
    breaklines=true,
    tabsize=2,
    prebreak=\raisebox{0ex}[0ex][0ex]{\ensuremath{\hookleftarrow}},
    frame=none,
    showtabs=false,
    showspaces=false,
    showstringspaces=false,
    prebreak={},
    keywordstyle=\color[rgb]{0.627,0.126,0.941},
    commentstyle=\color[rgb]{0.133,0.545,0.133},
    stringstyle=\color[rgb]{01,0,0},
    captionpos=t,
    escapeinside={(\%}{\%)}
}

\begin{document}

\begin{center}
    \section*{A - Amanvidia} % ganti judul soal

    \begin{tabular}{ | c c | }
        \hline
        Batas Waktu  & 2s \\    % jangan lupa ganti time limit
        Batas Memori & 512MB \\  % jangan lupa ganti memory limit
        \hline
    \end{tabular}
\end{center}

\subsection*{Deskripsi}
Sebuah bilangan dikatakan \textbf{Aman} jika dan hanya jika setiap digit bilangan ini hanya memakai digit 0, 1, 8, dan/atau 9.  Anda diminta mencari sebuah bilangan asli dengan banyak digit maksimal 25 yang merupakan kelipatan $N$ dan merupakan bilangan \textbf{Aman}.

\subsection*{Format Masukan}

Baris pertama berisi sebuah bilangan $N$ $(1 \leq N \leq 2 \times 10^7)$.


\subsection*{Format Keluaran}
Bilangan kelipatan $N$ yang merupakan suatu bilangan \textbf{Aman}. Jika tidak dapat dibuat menjadi bilangan \textbf{Aman}, keluarkan \lstinline|'-1'| (tanpa tanda petik).

\begin{multicols}{2}
\subsection*{Contoh Masukan 1}
\begin{lstlisting}
13
\end{lstlisting}
\columnbreak
\subsection*{Contoh Keluaran 1}
\begin{lstlisting}
91
\end{lstlisting}
\vfill
\null
\end{multicols}

\begin{multicols}{2}
\subsection*{Contoh Masukan 2}
\begin{lstlisting}
5
\end{lstlisting}
\columnbreak
\subsection*{Contoh Keluaran 2}
\begin{lstlisting}
10000
\end{lstlisting}
\vfill
\null
\end{multicols}

\subsection*{Penjelasan}
91 merupakan kelipatan dari 13, dan 10000 merupakan kelipatan dari 5.

\end{document}