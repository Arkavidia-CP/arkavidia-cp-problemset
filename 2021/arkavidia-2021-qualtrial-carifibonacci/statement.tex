\documentclass{article}

\usepackage{geometry}
\usepackage{amsmath}
\usepackage{graphicx, eso-pic}
\usepackage{listings}
\usepackage{hyperref}
\usepackage{multicol}
\usepackage{fancyhdr}
\pagestyle{fancy}
\fancyhf{}
\hypersetup{ colorlinks=true, linkcolor=black, filecolor=magenta, urlcolor=cyan}
\geometry{ a4paper, total={170mm,257mm}, top=40mm, right=20mm, bottom=20mm, left=20mm}
\setlength{\parindent}{0pt}
\setlength{\parskip}{0.3em}
\renewcommand{\headrulewidth}{0pt}
\AddToShipoutPictureBG{%
  \AtPageUpperLeft{%
    \raisebox{-\height}{\includegraphics[width=\paperwidth, height=30mm]{../headerarkav.png}}
  }
}
\rfoot{\thepage}
\lfoot{Warmup Penyisihan Competitive Programming - Arkavidia 7.0}
\lstset{
    basicstyle=\ttfamily\small,
    columns=fixed,
    extendedchars=true,
    breaklines=true,
    tabsize=2,
    prebreak=\raisebox{0ex}[0ex][0ex]{\ensuremath{\hookleftarrow}},
    frame=none,
    showtabs=false,
    showspaces=false,
    showstringspaces=false,
    prebreak={},
    keywordstyle=\color[rgb]{0.627,0.126,0.941},
    commentstyle=\color[rgb]{0.133,0.545,0.133},
    stringstyle=\color[rgb]{01,0,0},
    captionpos=t,
    escapeinside={(\%}{\%)}
}

\begin{document}

\begin{center}
    \section*{C - Cari Fibonacci}

    \begin{tabular}{ | c c | }
        \hline
        Batas Waktu  & 1s \\
        Batas Memori & 256MB \\
        \hline
    \end{tabular}
\end{center}

\subsection*{Deskripsi}

Barisan fibonacci merupakan suatu barisan bilangan bulat dimana setiap suku fibonacci merupakan penjumlahan dua suku sebelumnya. Secara matematis, bilangan fibonacci didefinisikan sebagai berikut:

\begin{align*}
F_1 & = 1 \\
F_2 & = 1 \\
F_n & = F_{n-1} + F_{n-2}
\end{align*}

Diberikan sebuah bilangan bulat $N$, tentukan nilai $F_N$ modulo $1.000.000.007$.

\subsection*{Format Masukan}

Baris pertama terdiri dari satu bilangan bulat positif $T$ ($1 \leq T \leq 100.000$), menyatakan banyaknya kasus uji.
Tiap kasus uji terdiri dari satu baris berisikan bilangan $N$ ($1 \leq N \leq 100.000$).

\subsection*{Format Keluaran}

Untuk tiap kasus uji, tuliskan sebuah baris berisi nilai dari $F_N$, dimodulo 1.000.000.007.
\\

\begin{multicols}{2}
\subsection*{Contoh Masukan}
\begin{lstlisting}
2
1
5
\end{lstlisting}
\columnbreak
\subsection*{Contoh Keluaran}
\begin{lstlisting}
1
5
\end{lstlisting}
\vfill
\null
\end{multicols}

\pagebreak

\end{document}