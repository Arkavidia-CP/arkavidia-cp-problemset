\documentclass{article}

\usepackage{geometry}
\usepackage{amsmath}
\usepackage{graphicx, eso-pic}
\usepackage{listings}
\usepackage{hyperref}
\usepackage{multicol}
\usepackage{fancyhdr}
\pagestyle{fancy}
\fancyhf{}
\hypersetup{ colorlinks=true, linkcolor=black, filecolor=magenta, urlcolor=cyan}
\geometry{ a4paper, total={170mm,257mm}, top=40mm, right=20mm, bottom=20mm, left=20mm}
\setlength{\parindent}{0pt}
\setlength{\parskip}{0.3em}
\renewcommand{\headrulewidth}{0pt}
\AddToShipoutPictureBG{%
  \AtPageUpperLeft{%
    \raisebox{-\height}{\includegraphics[width=\paperwidth, height=30mm]{../headerarkav.png}}
  }
}
\rfoot{\thepage}
\lfoot{Final Competitive Programming - Arkavidia 7.0}
\lstset{
    basicstyle=\ttfamily\small,
    columns=fixed,
    extendedchars=true,
    breaklines=true,
    tabsize=2,
    prebreak=\raisebox{0ex}[0ex][0ex]{\ensuremath{\hookleftarrow}},
    frame=none,
    showtabs=false,
    showspaces=false,
    showstringspaces=false,
    prebreak={},
    keywordstyle=\color[rgb]{0.627,0.126,0.941},
    commentstyle=\color[rgb]{0.133,0.545,0.133},
    stringstyle=\color[rgb]{01,0,0},
    captionpos=t,
    escapeinside={(\%}{\%)}
}

\begin{document}

\begin{center}
    \section*{I - Indra dan Senjata} % ganti judul soal

    \begin{tabular}{ | c c | }
        \hline
        Batas Waktu  & 2s \\    % jangan lupa ganti time limit
        Batas Memori & 1024MB \\  % jangan lupa ganti memory limit
        \hline
    \end{tabular}
\end{center}

\subsection*{Deskripsi}
Indra bosan mengurus pizza, kini Indra hanya ingin  mengurus senjata. Terdapat $N$ senjata, yang dinomori dari 1 sampai $N$, dan diletakkan secara berbaris. Senjata ke-$i$ memiliki kekuatan $K_i$. Indra ingin melakukan beberapa operasi sebagai berikut:

\begin{enumerate}

\item Mencari jumlah total kekuatan senjata dari $L$ sampai $R$ $(1 \leq L \leq R \leq N)$, yang kekuatannya berada pada range $\left[P, Q\right]$ $(P \leq Q)$, informasi ini digunakan untuk persiapan dalam mengambil senjata.

\item Mengganti senjata ke-$X$ dengan senjata baru yang memiliki kekuatan $V$

\end{enumerate}

Karena senjata yang ada sangat banyak, Indra meminta bantuan Anda untuk melakukan operasi-operasi tersebut, bantulah Indra!

\subsection*{Format Masukan}

Baris pertama berisi sebuah bilangan bulat positif $N$ $(1 \leq N \leq 10^5)$, menyatakan banyaknya senjata

Baris kedua berisi $N$ bilangan bulat $K_i$ $(1 \leq K_i \leq 10^5)$, menyatakan kekuatan senjata ke-$i$, untuk setiap $1 \leq i \leq N$.

Baris ketiga berisi sebuah bilangan bulat positif $Q$ $(1 \leq Q \leq 2 \times 10^4)$, menyatakan banyaknya operasi

$Q$ baris selanjutnya berisi salah satu dari dua macam operasi sebagai berikut:

\begin{itemize}

\item $1$ $L$ $R$ $P$ $Q$ $(1 \leq L \leq R \leq N, 1 \leq P \leq Q \leq 10^5)$

Menyatakan untuk mencari jumlah total dari $K_i$ dengan $(L \leq i \leq R)$ dengan ketentuan $(P \leq K_i \leq Q)$

\item $2$ $X$ $V$ $(1 \leq X \leq N, 1 \leq V \leq 10^5)$

Mengganti senjata ke-$X$ dengan senjata baru yang memiliki kekuatan $V$

\end{itemize}

\textbf{Catatan}:

Dijamin terdapat minimal satu buah operasi bentuk pertama

\subsection*{Format Keluaran}

Untuk setiap operasi pertama, keluarkan sebuah baris berisi sebuah bilangan bulat, jawaban dari operasi tersebut

\begin{multicols}{2}
\subsection*{Contoh Masukan}
\begin{lstlisting}
10
5 4 9 3 4 9 5 14 13 21
7
1 1 10 10 20
2 10 20
1 1 10 10 20
1 2 5 3 5
2 4 10
1 1 5 5 10
1 2 3 10 15

\end{lstlisting}
\columnbreak
\subsection*{Contoh Keluaran}
\begin{lstlisting}
27
47
11
24
0
\end{lstlisting}
\vfill
\null
\end{multicols}

\pagebreak

\subsection*{Penjelasan}

Pada awalnya kekuatan senjata adalah $\left[5, 4, 9, 3, 4, 9, 5, 14, 13, 21\right]$

\begin{itemize}

\item Pada operasi pertama, senjata yang memiliki range kekuatan [10, 20] dari senjata ke-1 sampai senjata ke-10 adalah senjata ke-8 dan senjata ke-9, sehingga totalnya adalah $$K_8 + K_9 = 14 + 13 = 27$$
\item Pada operasi kedua, kekuatan senjata ke-10 menjadi senjata baru dengan kekuatan = 20, kini kekuatan senjata menjadi $\left[5, 4, 9, 3, 4, 9, 5, 14, 13, 20\right]$
\item Pada operasi ketiga, senjata yang memiliki range kekuatan [10, 20] dari senjata ke-1 sampai senjata ke-10 adalah senjata ke-8, senjata ke-9, dan senjata ke-10, sehingga totalnya adalah

$$K_8 + K_9 + K_{10} = 14 + 13 + 20 = 47$$

\item Pada operasi keempat, senjata yang memiliki range kekuatan [3, 5] dari senjata ke-2 sampai senjata ke-5 adalah senjata ke-2, senjata ke-4, dan senjata ke-5, sehingga totalnya adalah

$$K_2 + K_4 + K_5 = 4 + 3 + 4 = 11$$

\item Pada operasi kelima, kekuatan senjata ke-4 menjadi senjata baru dengan kekuatan = 10, kini kekuatan senjata menjadi $\left[5, 4, 9, 10, 4, 9, 5, 14, 13, 20\right]$
\item Pada operasi keenam, senjata yang memiliki range kekuatan [5, 10] dari senjata ke-1 sampai senjata ke-5 adalah senjata ke-1, senjata ke-3, dan senjata ke-4, sehingga totalnya adalah

$$K_1 + K_3 + K_4 = 5 + 9 + 10 = 24$$

\item Pada operasi ketujuh, tidak ada kekuatan senjata yang memenuhi range, sehingga totalnya adalah 0

\end{itemize}

\pagebreak

\end{document}