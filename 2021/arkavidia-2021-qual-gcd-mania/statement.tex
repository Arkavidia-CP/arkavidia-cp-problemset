\documentclass{article}

\usepackage{geometry}
\usepackage{amsmath}
\usepackage{graphicx, eso-pic}
\usepackage{listings}
\usepackage{hyperref}
\usepackage{multicol}
\usepackage{fancyhdr}
\pagestyle{fancy}
\fancyhf{}
\hypersetup{ colorlinks=true, linkcolor=black, filecolor=magenta, urlcolor=cyan}
\geometry{ a4paper, total={170mm,257mm}, top=40mm, right=20mm, bottom=20mm, left=20mm}
\setlength{\parindent}{0pt}
\setlength{\parskip}{0.3em}
\renewcommand{\headrulewidth}{0pt}
\AddToShipoutPictureBG{%
  \AtPageUpperLeft{%
    \raisebox{-\height}{\includegraphics[width=\paperwidth, height=30mm]{../headerarkav.png}}
  }
}
\rfoot{\thepage}
\lfoot{Penyisihan Competitive Programming - Arkavidia 7.0}
\lstset{
    basicstyle=\ttfamily\small,
    columns=fixed,
    extendedchars=true,
    breaklines=true,
    tabsize=2,
    prebreak=\raisebox{0ex}[0ex][0ex]{\ensuremath{\hookleftarrow}},
    frame=none,
    showtabs=false,
    showspaces=false,
    showstringspaces=false,
    prebreak={},
    keywordstyle=\color[rgb]{0.627,0.126,0.941},
    commentstyle=\color[rgb]{0.133,0.545,0.133},
    stringstyle=\color[rgb]{01,0,0},
    captionpos=t,
    escapeinside={(\%}{\%)}
}

\begin{document}

\begin{center}
    \section*{G - GCD Mania} % ganti judul soal

    \begin{tabular}{ | c c | }
        \hline
        Batas Waktu  & 2s \\    % jangan lupa ganti time limit
        Batas Memori & 256MB \\  % jangan lupa ganti memory limit
        \hline
    \end{tabular}
\end{center}

\subsection*{Deskripsi}
Gaia seorang pengidap GCD Mania gemar menghitung GCD (Greatest Common Divisor).

Suatu hari, Gaia ditantang oleh Terra dalam menghitung GCD. Gaia diberi $N$ buah bilangan bulat $A_1, \dots, A_N$ oleh Terra. Ia lalu diberikan kuis dengan $Q$ pertanyaan. Apabila berhasil menjawab semua pertanyaan tersebut, Gaia akan diberikan hadiah oleh Terra. Untuk setiap pertanyaan, Gaia diberikan bilangan bulat $L$ dan $R$ $(L \leq R)$, kemudian Gaia diminta untuk mencari suatu bilangan bulat $X$ yang terletak pada rentang $[L, R]$, sedemikian sehingga:

\begin{center}
$GCD(A_1 + X, A_2 + X, \dots, A_N + X)$ bernilai maksimal
\end{center} 

Gaia perlu menjawab pertanyaan dengan memberikan nilai maksimal GCD tersebut, namun tidak perlu memberi tahu nilai $X$ kepada Terra. Gaia merasa soal ini terlalu sulit dan meminta bantuan Anda untuk mendapatkan hadiah Terra. Bantulah Gaia untuk mendapatkan hadiah tersebut!

\subsection*{Format Masukan}
Baris pertama berisi bilangan bulat $N$ $(1 \leq N \leq 10^5)$, menyatakan banyaknya bilangan bulat yang diberikan Terra.

Baris kedua berisi $N$ buah bilangan bulat $A_i$ $(1 \leq A_i \leq 10^7)$ untuk setiap $1 \leq i \leq N$, menyatakan nilai dari $N$ bilangan bulat yang diberikan Terra.

Baris berikutnya berisi sebuah bilangan bulat $Q$ $(1 \leq Q \leq 10^5)$, menyatakan banyaknya pertanyaan.

$Q$ baris berikutnya berisi dua buah bilangan $L_i$ dan $R_i$, $(0 \leq L_i \leq R_i, \leq 10^7)$ untuk setiap $i$ $(1 \leq i \leq Q)$.

\subsection*{Format Keluaran}
Keluarkan $Q$ baris, dengan setiap baris merupakan jawaban dari pertanyaan Terra.

\begin{multicols}{2}
\subsection*{Contoh Masukan}
\begin{lstlisting}
3
1 5 9
2
1 2
2 10
\end{lstlisting}
\columnbreak
\subsection*{Contoh Keluaran}
\begin{lstlisting}
2
4
\end{lstlisting}
\vfill
\null
\end{multicols}


\subsection*{Penjelasan}

Pada query pertama, cara optimal adalah dengan memilih $X = 1$, sehingga didapat:
$$GCD(1+1,5+1,9+1) = GCD(2,6,10) = 2$$

Pada query kedua, salah satu cara optimal adalah dengan memilih $X = 3$, sehingga didapat:
$$GCD(1+3,5+3,9+3) = GCD(4,8,12) = 4$$

\pagebreak

\end{document}