\documentclass{article}

\usepackage{geometry}
\usepackage{amsmath}
\usepackage{graphicx, eso-pic}
\usepackage{listings}
\usepackage{hyperref}
\usepackage{multicol}
\usepackage{fancyhdr}
\pagestyle{fancy}
\fancyhf{}
\hypersetup{ colorlinks=true, linkcolor=black, filecolor=magenta, urlcolor=cyan}
\geometry{ a4paper, total={170mm,257mm}, top=40mm, right=20mm, bottom=20mm, left=20mm}
\setlength{\parindent}{0pt}
\setlength{\parskip}{0.5em}
\renewcommand{\headrulewidth}{0pt}
\AddToShipoutPictureBG{%
  \AtPageUpperLeft{%
    \raisebox{-\height}{\includegraphics[width=\paperwidth, height=30mm]{../headerarkav.png}}
  }
}
\rfoot{\thepage}
\lfoot{Competitive Programming - Arkavidia 8.0}
\lstset{
    basicstyle=\ttfamily\small,
    columns=fixed,
    extendedchars=true,
    breaklines=true,
    tabsize=2,
    prebreak=\raisebox{0ex}[0ex][0ex]{\ensuremath{\hookleftarrow}},
    frame=none,
    showtabs=false,
    showspaces=false,
    showstringspaces=false,
    prebreak={},
    keywordstyle=\color[rgb]{0.627,0.126,0.941},
    commentstyle=\color[rgb]{0.133,0.545,0.133},
    stringstyle=\color[rgb]{01,0,0},
    captionpos=t,
    escapeinside={(\%}{\%)}
}

\begin{document}

\begin{center}
    \section*{J. J-cantik} % ganti judul soal

    \begin{tabular}{ | c c | }
        \hline
        Batas Waktu  & 2s \\    % jangan lupa ganti time limit
        Batas Memori & 512MB \\  % jangan lupa ganti memory limit
        \hline
    \end{tabular}
\end{center}

\subsection*{Deskripsi}
Pada suatu hari, Arka sedang berjalan-jalan di sekolahnya sembari menunggu jadwal kelasnya. Ia lalu menemukan sebuah barisan $A$ yang berisi $N$ bilangan asli. Karena sedang bosan, Arka memutuskan untuk bermain-main dengan barisan bilangan tersebut. Arka ingin menghubungkan barisan tersebut dengan bilangan favoritnya, yaitu $J$. 

Setelah berpikir cukup lama, Arka akhirnya memutuskan untuk mencari banyaknya \textit{pair} J-cantik yang ada pada barisan $A$. Sebuah \textit{pair} $(k,l)$ disebut \textit{pair} \textbf{J-cantik} jika banyaknya bilangan prima yang habis membagi $A_k \times$ $A_{k+1}$ $\times \cdots \times A_l$ ($1 \leq$ $k$ $\leq  l$  $\leq  N$) adalah $J$. Arka menyadari bahwa mencari banyaknya \textit{pair} J-cantik bukanlah hal yang mudah, terutama waktu Arka untuk menemukannya sangatlah terbatas karena waktu masuk kelas sudah sebentar lagi. Bantu Arka menemukan banyaknya \textit{pair} J-cantik yang memenuhi.

\subsection*{Format Masukan}
Baris pertama terdiri dari dua buah bilangan bulat $N$ dan $J$ ($1 \leq$ $N$ $\leq 10^5;$ $0 \leq$ $J$ $\leq 10^5$) yang masing-masing menyatakan banyaknya bilangan pada barisan $A$ dan bilangan favorit Arka.

Baris kedua terdiri dari $N$ bilangan asli $A_1$, $A_2$, $A_3$, \cdots, $A_N$ ($1 \leq$ $A_i$ $ \leq 10^5$) yang masing-masing menyatakan bilangan ke-$i$ pada barisan $A$.

\subsection*{Format Keluaran}
Tuliskan banyaknya \textit{pair} J-cantik berbeda yang memenuhi.

\begin{multicols}{2}
\subsection*{Contoh Masukan 1}
\begin{lstlisting}
4 2
6 2 3 5
\end{lstlisting}
\columnbreak

\subsection*{Contoh Keluaran 1}
\begin{lstlisting}
5
\end{lstlisting}
\vfill
\null
\end{multicols}

\begin{multicols}{2}
\subsection*{Contoh Masukan 2}
\begin{lstlisting}
5 0
1 2 1 1 3
\end{lstlisting}
\columnbreak

\subsection*{Contoh Keluaran 2}
\begin{lstlisting}
4
\end{lstlisting}
\vfill
\null
\end{multicols}

\begin{multicols}{2}
\subsection*{Contoh Masukan 3}
\begin{lstlisting}
6 1
2 3 5 4 7 6
\end{lstlisting}
\columnbreak

\subsection*{Contoh Keluaran 3}
\begin{lstlisting}
5
\end{lstlisting}
\vfill
\null
\end{multicols}

\subsection*{Penjelasan}
Pada contoh masukan pertama, terdapat 5 \textit{pair} J-cantik yang memenuhi yaitu:
\begin{itemize}
    \item $(1,1)$
    \item $(1,2)$
    \item $(1,3)$
    \item $(2,3)$
    \item $(3,4)$
\end{itemize}

\end{document}