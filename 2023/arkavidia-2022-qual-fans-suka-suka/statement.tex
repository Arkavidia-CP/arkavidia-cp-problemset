\documentclass{article}

\usepackage{geometry}
\usepackage{amsmath}
\usepackage{graphicx, eso-pic}
\usepackage{listings}
\usepackage{hyperref}
\usepackage{multicol}
\usepackage{fancyhdr}
\pagestyle{fancy}
\fancyhf{}
\hypersetup{ colorlinks=true, linkcolor=black, filecolor=magenta, urlcolor=cyan}
\geometry{ a4paper, total={170mm,257mm}, top=40mm, right=20mm, bottom=20mm, left=20mm}
\setlength{\parindent}{0pt}
\setlength{\parskip}{0.5em}
\renewcommand{\headrulewidth}{0pt}
\AddToShipoutPictureBG{%
  \AtPageUpperLeft{%
    \raisebox{-\height}{\includegraphics[width=\paperwidth, height=30mm]{../headerarkav.png}}
  }
}
\rfoot{\thepage}
\lfoot{Competitive Programming - Arkavidia 8.0}
\lstset{
    basicstyle=\ttfamily\small,
    columns=fixed,
    extendedchars=true,
    breaklines=true,
    tabsize=2,
    prebreak=\raisebox{0ex}[0ex][0ex]{\ensuremath{\hookleftarrow}},
    frame=none,
    showtabs=false,
    showspaces=false,
    showstringspaces=false,
    prebreak={},
    keywordstyle=\color[rgb]{0.627,0.126,0.941},
    commentstyle=\color[rgb]{0.133,0.545,0.133},
    stringstyle=\color[rgb]{01,0,0},
    captionpos=t,
    escapeinside={(\%}{\%)}
}

\begin{document}

\begin{center}
    \section*{F. Fans Suka-Suka} % ganti judul soal

    \begin{tabular}{ | c c | }
        \hline
        Batas Waktu  & 1s \\    % jangan lupa ganti time limit
        Batas Memori & 256MB \\  % jangan lupa ganti memory limit
        \hline
    \end{tabular}
\end{center}

\subsection*{Deskripsi}
Turnamen \textit{The Nasional} telah dimulai!

Turnamen \textit{The Nasional} diikuti oleh $2^N$ tim yang dinomori $1$ sampai $2^N$. Peringkat kekuatan setiap tim dinyatakan dengan sekuens $K$. Tim bernomor $i$ memiliki peringkat kekuatan $K_i$. Tim bernomor $i$ dianggap lebih kuat dari tim bernomor $j$ apabila $K_i < K_j$. Sekuens $K$ merupakan permutasi bilangan bulat $1$ sampai $2^N$ sehingga tidak ada dua tim yang memiliki peringkat kekuatan yang sama.

Turnamen diadakan dalam format \textit{single elimination} sehingga total babak dalam turnamen ada sebanyak N babak. Babak ke-$i$ berjalan seperti berikut.
\begin{enumerate}
    \item Tim-tim yang belum gugur diurutkan berdasarkan \textbf{nomor tim} secara menaik.
    \item Untuk setiap $j$ dari $1$ sampai $2^{N-i}$, tim urutan ke-$(2j-1)$ dan $2j$ bertanding. Perhatikan bahwa tim yang lebih kuat \textbf{belum} tentu menang.
\end{enumerate}

\textit{The Nasional} kali ini terasa berbeda dengan adanya fenomena \textit{Fans SukaSuka}. Dalam fenomena ini, fans setiap tim $i$ terbagi menjadi dua jenis:
\begin{enumerate}
    \item fans yang hanya akan menonton pertandingan apabila tim kesukaannya lebih kuat dari tim lawan sebanyak $A_i$, dan
    \item fans yang hanya akan menonton pertandingan apabila tim kesukaannya lebih lemah dari tim lawan sebanyak $B_i$. 
\end{enumerate}
Akibatnya, dalam pertandingan antara tim bernomor $i$ dan tim bernomor $j$, apabila $K_i < K_j$, terdapat $A_i + B_j$ penonton.

Arka, penggemar setia \textit{The Nasional}, ingin menghitung penonton total \textit{The Nasional} maksimum yang mungkin terjadi. Bantulah dia menentukan pentonton total maksimum yang mungkin!

\subsection*{Format Masukan}
Baris pertama masukan berisi sebuah bilangan bulat $N$ $(1 \leq N \leq 17)$. Nilai $2^N$ menyatakan banyak tim dalam \textit{The Nasional}.

Masukan dilanjutkan dengan $2^N$ baris. Baris ke-$i$ berisi tiga bilangan bulat $K_i$, $A_i$, dan $B_i$ $(1 \leq K_i \leq 2^N, 1 \leq A_i, B_i \leq 10^9)$ yang menyatakan peringkat kekuatan, banyak fans jenis pertama, dan banyak fans jenis kedua dari tim bernomor $i$. Dijamin bahwa setiap bilangan bulat dari $1$ sampai $2^N$ muncul tepat satu kali di sekuens $K$.

\subsection*{Format Keluaran}
Keluaran terdiri atas satu baris berisi sebuah bilangan bulat yang menyatakan penonton total \textit{The Nasional} maksimum yang mungkin terjadi.

\pagebreak
\begin{multicols}{2}
\subsection*{Contoh Masukan 1}
\begin{lstlisting}
2
1 2 1
3 4 4
4 2 1
2 4 1
\end{lstlisting}
\null
\columnbreak

\subsection*{Contoh Keluaran 1}
\begin{lstlisting}
19
\end{lstlisting}
\vfill
\null
\end{multicols}

\begin{multicols}{2}
\subsection*{Contoh Masukan 2}
\begin{lstlisting}
1
2 1 1
1 1 1
\end{lstlisting}
\null
\columnbreak

\subsection*{Contoh Keluaran 2}
\begin{lstlisting}
2
\end{lstlisting}
\vfill
\null
\end{multicols}

\begin{multicols}{2}
\subsection*{Contoh Masukan 3}
\begin{lstlisting}
3
5 5 7
3 5 1
2 1 9
1 4 7
6 5 9
7 8 2
8 9 8
4 1 2
\end{lstlisting}
\null
\columnbreak

\subsection*{Contoh Keluaran 3}
\begin{lstlisting}
81
\end{lstlisting}
\vfill
\null
\end{multicols}

\subsection*{Penjelasan}
Pada testcase pertama, salah satu kemungkinan alur turnamen yang menghasilkan penonton total maksimum adalah seperti berikut.
\begin{enumerate}
    \item Pada babak pertama, urutan tim berdasarkan nomor adalah $[1,2,3,4]$. Di babak ini terdapat dua pertandingan berikut.
    \begin{itemize}
        \item Tim bernomor $1$ melawan tim bernomor $2$ dengan $A_1 + B_2 = 2 + 4 = 6$ penonton. Tim bernomor $2$ memenangkan pertandingan ini.
        \item Tim bernomor $3$ melawan tim bernomor $4$ dengan $B_3 + A_4 = 1 + 4 = 5$ penonton. Tim bernomor $4$ memenangkan pertandingan ini.
    \end{itemize}
    \item Pada babak kedua, urutan tim berdasarkan nomor adalah $[2,4]$. Di babak ini terdapat satu pertandingan berikut.
    \begin{itemize}
        \item Tim bernomor $2$ melawan tim bernomor $4$ dengan $B_2 + A_4 = 4 + 4 = 8$ penonton. Tim bernomor $2$ memenangkan pertandingan ini.
    \end{itemize}
\end{enumerate}
Terlihat bahwa penonton total kemungkinan di atas adalah $19$. Dapat ditunjukkan bahwa tidak ada kemungkinan alur turnamen yang menghasilkan penonton total yang lebih besar dari $19$.

Pada testcase kedua, hanya terdapat satu pertandingan di turnamen, yaitu pertandingan dengan $2$ penonton.

\end{document}