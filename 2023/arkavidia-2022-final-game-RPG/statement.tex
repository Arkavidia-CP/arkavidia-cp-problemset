\documentclass{article}

\usepackage{geometry}
\usepackage{amsmath}
\usepackage{graphicx, eso-pic}
\usepackage{listings}
\usepackage{hyperref}
\usepackage{multicol}
\usepackage{fancyhdr}
\pagestyle{fancy}
\fancyhf{}
\hypersetup{ colorlinks=true, linkcolor=black, filecolor=magenta, urlcolor=cyan}
\geometry{ a4paper, total={170mm,257mm}, top=40mm, right=20mm, bottom=20mm, left=20mm}
\setlength{\parindent}{0pt}
\setlength{\parskip}{0.3em}
\renewcommand{\headrulewidth}{0pt}
\AddToShipoutPictureBG{%
  \AtPageUpperLeft{%
    \raisebox{-\height}{\includegraphics[width=\paperwidth, height=30mm]{../headerarkav.png}}
  }
}
\rfoot{\thepage}
\lfoot{Competitive Programming - Arkavidia 8.0}
\lstset{
    basicstyle=\ttfamily\small,
    columns=fixed,
    extendedchars=true,
    breaklines=true,
    tabsize=2,
    prebreak=\raisebox{0ex}[0ex][0ex]{\ensuremath{\hookleftarrow}},
    frame=none,
    showtabs=false,
    showspaces=false,
    showstringspaces=false,
    prebreak={},
    keywordstyle=\color[rgb]{0.627,0.126,0.941},
    commentstyle=\color[rgb]{0.133,0.545,0.133},
    stringstyle=\color[rgb]{01,0,0},
    captionpos=t,
    escapeinside={(\%}{\%)}
}

\begin{document}

\begin{center}
    \section*{G. Game RPG} % ganti judul soal

    \begin{tabular}{ | c c | }
        \hline
        Batas Waktu  & 1s \\    % jangan lupa ganti time limit
        Batas Memori & 256MB \\  % jangan lupa ganti memory limit
        \hline
    \end{tabular}
\end{center}

\subsection*{Deskripsi}
Pada suatu hari, Arvy sedang bermain suatu game RPG bernama \textit{Spire of Fantasy}. Ia berhasil menemukan suatu menara kuno yang berisi harta karun. Menara tersebut memiliki $N$ lantai bernomor 1 sampai $N$. Masing-masing lantai memiliki $M$ ruangan bernomor 1 sampai $M$ yang tersusun melingkar. Pada ruangan ke-$j$ di lantai ke-$i$, terdapat $V_{i,j}$ buah keping emas. Ruangan bernomor $j$ pada lantai bernomor $i$ tersebut terhubung dengan ruangan bernomor ($j$-2) (mod $M$) + 1, ($j$-1) (mod $M$) + 1, dan $j$ (mod $M$) + 1 pada lantai bernomor $i+1$ ($1 \leq$ $i$ $<$ $N$). Ruangan pada lantai teratas semuanya terhubung ke satu jalan keluar. 

Arvy mulai mengambil harta dari lantai 1 dan ia dapat memilih untuk memulai dari ruangan manapun pada lantai 1. Jika Arvy memasuki suatu ruangan, ia harus mengambil semua keping emas yang ada di dalam ruangan tersebut. Setelah Arvy naik ke lantai berikutnya, ia tidak bisa turun ke lantai sebelumnya. Karena Arvy merupakan orang yang ambisius, ia ingin mengambil harta sebanyak mungkin. 

Saat Arvy hendak memulai perjalanannya, Arka datang dan memberi tantangan kepada Arvy. Jika Arvy berhasil mengambil tepat $X$ keping emas, Arka akan memberikan hadiah tambahan sebesar $Y$ keping emas. Bantulah Arvy menentukan jalur optimum untuk mendapatkan harta terbanyak.

\subsection*{Format Masukan}
Baris pertama terdiri dari empat buah bilangan bulat $N$, $M$, $X$, dan $Y$  ($3 \leq$ $N$, $M$ $\leq 20$, $1 \leq$ $X$, $Y$ $\leq 10^{10}$) yang masing-masing merepresentasikan banyaknya lantai, banyaknya ruangan pada setiap lantai, banyak keping emas yang ditargetkan Arka, dan banyaknya hadiah keping emas yang akan diberikan Arka jika berhasil menyelesaikan tantangan.

$N$ baris berikutnya terdiri dari $M$ buah bilangan $V_{i,1}$, $V_{i,2}$, \cdots, $V_{i,M}$ ($1 \leq$ $V_{i,j}$ $ \leq 10^{10}$) yang menyatakan banyak keping emas di lantai ke-$i$ pada ruangan ke-$j$.

\subsection*{Format Keluaran}
Baris pertama berisi sebuah bilangan bulat yang menyatakan banyak keping emas maksimum yang dapat diperoleh Arvy.

Baris kedua berisi $N$ bilangan yang dipisahkan dengan spasi, yang masing-masing bilangan menyatakan ruangan yang harus dipilih pada lantai ke-$i$. Jika ada banyak jalur yang dapat menghasilkan keping emas maksimum, keluarkan jalur yang \textit{lexicographically minimum}.

(Note: Jalur yang lexicographically minimum adalah jalur yang memiliki urutan ruangan yang paling kecil. Misalnya, jika ada dua jalur yang memiliki banyak keping emas yang sama, jalur yang lexicographically minimum adalah jalur yang memiliki urutan ruangan yang paling kecil. Contoh: $1 2 3$ lebih kecil dari $1 3 2$).

\begin{multicols}{2}
\subsection*{Contoh Masukan 1}
\begin{lstlisting}
3 3 5 5
1 3 1
2 2 1
1 2 1
\end{lstlisting}
\columnbreak

\subsection*{Contoh Keluaran 1}
\begin{lstlisting}
10
1 1 2
\end{lstlisting}
\vfill
\null
\end{multicols}

\begin{multicols}{2}
\subsection*{Contoh Masukan 2}
\begin{lstlisting}
3 3 4 2
1 3 1
2 2 1
1 2 3
\end{lstlisting}
\columnbreak

\subsection*{Contoh Keluaran 2}
\begin{lstlisting}
8
2 1 3
\end{lstlisting}
\vfill
\null
\end{multicols}

\end{document}